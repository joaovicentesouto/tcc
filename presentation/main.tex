% Ao menos uma linguagem (brazil ou english) deveria sempre ser fornecida
\documentclass[english]{lapesd-slides}

\usepackage{pgfgantt}

%%%%%%%%%%%%%%%%%%%%%%%%%%%%
% Metadados
%%%%%%%%%%%%%%%%%%%%%%%%%%%%

\title[An Inter-Cluster Comm. Facility for LMP in the Nanvix OS]{
	An Inter-Cluster Communication Facility for Lightweight Manycore Processors in the Nanvix OS
}
% \subtitle{}
\author[J. V. Souto]{
	\large João Vicente Souto\\
	{\small \texttt{joao.vicente.souto@grad.ufsc.br}}
}

\institute{
	\fontsize{10.5}{12.6}\selectfont 
	Graduação em Ciência da Computação\\ 
	Depto. de Informática e Estatísitca\\
	Universidade Federal de Santa Catarina - Florianópolis\\
	\vspace{1em}
	\large Orientador: Prof. Márcio Bastos Castro, Dr.\\
	Coorientador: Pedro Henrique Penna, Me.
}

\date{\today}

%%%%%%%%%%%%%%%%%%%%%%%%%%%%
% Slides
%%%%%%%%%%%%%%%%%%%%%%%%%%%%

\begin{document}

\titleframe

% Você não é obrigado a colocar um sumário!
\begin{frame}{Sumário}
  \tableofcontents
\end{frame}

% Desse ponto em diante serão inseridos slides de pausa a cada \section
% \showsections
\section{Introdução}

\pholder[O que é um processador?]{Processadores}

\pholder[Como deixar os CPUs mais poderosos?]{Evolução dos Processadores}

\pholder[O que motivou o surgimento do Lightweight Manycore?]{Preocupações atuais}

\pholder[Quais são as características de um Lightweight Manycore?]{Lightweight Manycore}

\section{Objetivos}

\pholder[Quais são os objetivos do trabalho?]{Objetivos}

\section{Fundamentação}

\pholder[Quais conceitos são importantes para entender o trabalho?]{Fundamentação}

\section{Desenvolvimento}

\pholder[O que foi feito no nível da HAL?]{Low level}

\pholder[O que foi feito no nível do Microkernel?]{User level}

\section{Experimentos}

\pholder[Quais experimentos foram feitos?]{Experimentos}

\pholder[Quais foram os resultados?]{Resultados}

\section{Conclusões}

\pholder[Quais foram as conclusões?]{Conclusões}

%%%%%%%%%%%%%%%%%%%%%%%%%%%%
% Finalização
%%%%%%%%%%%%%%%%%%%%%%%%%%%%

\thanksframe
\referencesframe{references}

\begin{backup}
  \pholder{Slide de backup}
\end{backup}

\end{document}

% LocalWords:  template cls standalone GitHub Overleaf bugfixes SVGs
% LocalWords:  Re-empacotamento fontsize Makefile pdflatex imgs PDFs
% LocalWords:  shell-escape frames SVG brazil english lapesd-slides
% LocalWords:  disabletodonotes todonotes TODO's backup showbackup
% LocalWords:  hidebackup abntexcite abntex natbib nobib titleframe
% LocalWords:  frame showsections sidebar stopcountingframes default
% LocalWords:  thanksframe Thank You Questions referencesframe titulo
% LocalWords:  bibfiles pholder todonote placeholder inline addfig
% LocalWords:  opts graphicx addfiglw width Citations dijkstra Direct
% LocalWords:  Closure Parallel dynamic scheduling DoImportantStuff
% LocalWords:  lccp merged cell svg pdf
