% Ao menos uma linguagem (brazil ou english) deveria sempre ser fornecida
\documentclass[english]{lapesd-slides}

\usepackage{pgfgantt}

%%%%%%%%%%%%%%%%%%%%%%%%%%%%
% Metadados
%%%%%%%%%%%%%%%%%%%%%%%%%%%%

\title[An Inter-Cluster Comm. Facility for LMP in the Nanvix OS]{
	An Inter-Cluster Communication Facility for Lightweight Manycore Processors in the Nanvix OS
}
% \subtitle{}
\author[J. V. Souto]{
	\large João Vicente Souto\\
	{\small \texttt{joao.vicente.souto@grad.ufsc.br}}
}

\institute{
	\fontsize{10.5}{12.6}\selectfont 
	Graduação em Ciência da Computação\\ 
	Depto. de Informática e Estatísitca\\
	Universidade Federal de Santa Catarina - Florianópolis\\
	\vspace{1em}
	\large Orientador: Prof. Márcio Bastos Castro, Dr.\\
	Coorientador: Pedro Henrique Penna, Me.
}

\date{\today}

%%%%%%%%%%%%%%%%%%%%%%%%%%%%
% Slides
%%%%%%%%%%%%%%%%%%%%%%%%%%%%

\begin{document}

\titleframe

% Você não é obrigado a colocar um sumário!
\begin{frame}{Outline}
  \tableofcontents
\end{frame}

% Desse ponto em diante serão inseridos slides de pausa a cada \section
% \showsections
\section{Introduction}

	\pholder[O que é um processador?]{Processadores}

	% \pholder[Como deixar os CPUs mais poderosos?]{Evolução dos Processadores}

	\begin{frame}{Processor Trend}
		\begin{overprint}
			\only<1>{\includegraphics[width=.9\linewidth]{processor-trend-1.pdf}}
			\only<2>{\includegraphics[width=.9\linewidth]{processor-trend-2.pdf}}
			\only<3>{\includegraphics[width=.9\linewidth]{processor-trend-3.pdf}}
			\only<4>{\includegraphics[width=.9\linewidth]{processor-trend-4.pdf}}
			\only<5>{\includegraphics[width=.9\linewidth]{processor-trend-5.pdf}}
			\only<6>{\includegraphics[width=.9\linewidth]{processor-trend-6.pdf}}
		\end{overprint}
	\end{frame}

	\pholder[O que motivou o surgimento do Lightweight Manycore?]{Preocupações atuais}

	\pholder[Quais são as características de um Lightweight Manycore?]{Lightweight Manycore}

	\pholder[Qual o contexto do trabalho? Contexto de pesquisa sobre SO para manycores]{Contexto do Trabalho}

\section{Goals}

	\pholder[Quais são os objetivos do trabalho?]{Objetivos}

\section{Background}

	\subsection{Operating Systems}

		% \pholder[Quais modelos existem? Replicado, Mestre-Escravo e Compartilhado]{Multiprocessor Operating Systems}

		\begin{frame}[fragile]{Replicated OS}
			\addfiglw[texto 1]{imgs/replicated-os.pdf}
		\end{frame}

		\begin{frame}[fragile]{Master-Slave OS}
			\addfiglw[texto 1]{imgs/master-slave-os.pdf}
		\end{frame}

		\begin{frame}[fragile]{Symmetric OS}
			\addfiglw[texto 1]{imgs/symmetric-os.pdf}
		\end{frame}

	\subsection{Inter-Process Communication}

		\pholder[Hybrid Programming]{Inter-Process Communication}

		\pholder[Software que lida com interfaces e dma?]{Low level Communication}

		\pholder[Tipos de envio/recebimento?]{User level Communication}

	\subsection{MPPA-256}

		\pholder[Descrever MPPA]{MPPA-256}

	\subsection{Nanvix}

		\pholder{Nanvix OS}

		\pholder{Nanvix HAL}

		\pholder{Nanvix Microkernel}

		\pholder{Nanvix Multikernel}

\section{Development}

	\subsection{Low-Level Communication}

		\pholder[Interrup System and DMA mediator]{MPPA-256 Hardware Resources}

		\pholder{Noc Identifiers}

		\pholder{Resource Identifiers}

		\pholder{Hardware Limitations}

		\pholder[Lazy transfer and Interface convention]{General Concepts of Comm. Abstrations}

		\pholder[Concept]{Sync Abstration Concept}

		\pholder[Implementation]{Sync Abstration Implementation}

		\pholder[Concept]{Mailbox Abstration Concept}

		\pholder[Implementation]{Mailbox Abstration Implementation}

		\pholder[Concept]{Portal Abstration Concept}

		\pholder[Implementation]{Portal Abstration Implementation}

	\subsection{User-Level Communication}

		\pholder{Impacts of the Master-Slave Model}

		\pholder[Protection and Management, Multiplexing, Validation and Correctness Tests]{Details}

\section{Experiments}

	\pholder{Methodology}

	\pholder{Micro-benchmarks}

	\pholder{Design}

	\pholder{Portal Analysis}

	\pholder{Mailbox Analysis}

\section{Conclusions}

	\pholder[Quais foram as conclusões?]{Conclusions}

%%%%%%%%%%%%%%%%%%%%%%%%%%%%
% Finalização
%%%%%%%%%%%%%%%%%%%%%%%%%%%%

\thanksframe
\referencesframe{references}

\begin{backup}
  \pholder{Slide de backup}
\end{backup}

\end{document}

% LocalWords:  template cls standalone GitHub Overleaf bugfixes SVGs
% LocalWords:  Re-empacotamento fontsize Makefile pdflatex imgs PDFs
% LocalWords:  shell-escape frames SVG brazil english lapesd-slides
% LocalWords:  disabletodonotes todonotes TODO's backup showbackup
% LocalWords:  hidebackup abntexcite abntex natbib nobib titleframe
% LocalWords:  frame showsections sidebar stopcountingframes default
% LocalWords:  thanksframe Thank You Questions referencesframe titulo
% LocalWords:  bibfiles pholder todonote placeholder inline addfig
% LocalWords:  opts graphicx addfiglw width Citations dijkstra Direct
% LocalWords:  Closure Parallel dynamic scheduling DoImportantStuff
% LocalWords:  lccp merged cell svg pdf
