\begin{backup}

	\begin{frame}[fragile]{Network-on-Chip Identifiers}
		\begin{table}
			\centering%
			\begin{tabular}{l|l|l|}
				\cline{2-3}
													& \textbf{Physical ID} & \textbf{Logical ID} \\ \hline
				\multicolumn{1}{|l|}{\textbf{IO 0}} & 128-131              & 0-3                 \\ \hline
				\multicolumn{1}{|l|}{\textbf{IO 1}} & 192-195              & 4-7                 \\ \hline
				\multicolumn{1}{|l|}{\textbf{CCs}}  & 0-15                 & 8-23                \\ \hline
			\end{tabular}
		\end{table}

		% As interface NoC na HAL possuem dois identificadores.
		% O identificador físico é definido pelo próprio MPPA-256, o utilizando para rotear as comunicações através da NoC.
		% Para eliminar a dependência do identificador físico, foi criado um mapeamento para um identificador lógico, o qual é utilizado pelas abstrações fora da HAL.
	\end{frame}

	\begin{frame}[fragile]{Resource Identifiers}
		\begin{table}
			\centering%
			\begin{tabular}{l|l|l|l|l|}
				\cline{2-5}
														& \multicolumn{2}{c|}{\textbf{C-NoC}} & \multicolumn{2}{c|}{\textbf{D-NoC}} \\ \cline{2-5}
														& \textbf{RX ID} & \textbf{TX ID}     & \textbf{RX ID}  & \textbf{TX ID}    \\ \hline
				\multicolumn{1}{|l|}{\textbf{Mailbox}} & 0-23           & 0                  & 0-23            & 1-3               \\ \hline
				\multicolumn{1}{|l|}{\textbf{Portal}}  & 24-47          & 1-2                & 24-47           & 4-7               \\ \hline
				\multicolumn{1}{|l|}{\textbf{Sync}}    & 48-71          & 3                  & -               & -                 \\ \hline
			\end{tabular}
		\end{table}

		% Por outro lado, para prover transparência da comunicação das abstrações, é necessário que ela conheçam o recursos destino.
		% Para isso, os recursos de comunicação foram particionados por abstração.
		% Onde, dependendo dos nós envolvidos, é possível inferir os identificadores dos recursos.

		% %%%%%%Por exemplo, cada nó possui um slot de recebimento de dados na mailbox, permitindo que qualquer nó saiba o par interface, recurso que devem enviar os dados.
	\end{frame}

	\begin{frame}[fragile]{Simplified NoC handler algorithm}
			
		\begin{algorithm}
			% \caption{Simplified NoC handler algorithm.}%
			% \label{alg.noc-handler}%
			\begin{algorithmic}[1]
				\Require $status[M_{Interfaces}][N_{Resources}]$, interrupt status of a resource.
				\Require $handlers[M_{Interfaces}][N_{Resources}]$, interrupt handler of a resource.
				\Procedure{noc\_handler}{}
				\For {$i \in [1, M_{Interfaces}]$}
					\For {$j \in [1, N_{Resources}]$}
						\If {$status[i][j] == Interrupt Triggered$}
							\State {$\Call{clean\_status}{i, j}$}
							\State {$\Call{handlers[i][j]}{i, j}$}
						\EndIf
					\EndFor
				\EndFor
				\EndProcedure
			\end{algorithmic}%
			% \fonte{Developed by the author.}%
		\end{algorithm}

	\end{frame}

	\begin{frame}[fragile]{Simplified lazy transfer algorithm}
			
		\begin{algorithm}
  			\scriptsize
			% \caption{.}%
			% \label{alg:lazy-transfer}%
			\begin{algorithmic}[1]
			\Require $resources$, Abstraction Resource Table

			\Algphase{Configures data transfer.}

				\Procedure{async\_write}{$id, message, size$}
					\State {$resources[id].message \gets message$}
					\State {$resources[id].size \gets size$}
					\If {$resources[id].has\_permission$} 
						\State {$\Call{do\_lazy\_transfer}{id}$}
					\Else
						\State {$resources[id].is\_waiting \gets True$}
					\EndIf
				\EndProcedure%

			\Algphase{Receives permission.}

				\Procedure{abstraction\_handler}{$id$} 
					\If {$resources[id].is\_waiting$} 
						\State {$\Call{do\_lazy\_transfer}{id}$} 
					\Else
						\State {$resources[id].has\_permission \gets True$}
					\EndIf
				\EndProcedure%

			\Algphase{Transfers the data.}

				\Procedure{do\_lazy\_transfer}{$id$} 
					\State {$resources[id].is\_waiting \gets False$}
					\State {$resources[id].has\_permission \gets False$}
					\State {$\Call{transfer\_data}{resources[id].message, resources[id].size}$} 
					\State {$\Call{unlock}{resources[id].lock}$}                                \Comment{Releases slave core.}
				\EndProcedure%

			\end{algorithmic}%

			% \fonte{Developed by the author.}%
		\end{algorithm}

	\end{frame}

\end{backup}

% LocalWords:  template cls standalone GitHub Overleaf bugfixes SVGs
% LocalWords:  Re-empacotamento fontsize Makefile pdflatex imgs PDFs
% LocalWords:  shell-escape frames SVG brazil english lapesd-slides
% LocalWords:  disabletodonotes todonotes TODO's backup showbackup
% LocalWords:  hidebackup abntexcite abntex natbib nobib titleframe
% LocalWords:  frame showsections sidebar stopcountingframes default
% LocalWords:  thanksframe Thank You Questions referencesframe titulo
% LocalWords:  bibfiles pholder todonote placeholder inline addfig
% LocalWords:  opts graphicx addfiglw width Citations dijkstra Direct
% LocalWords:  Closure Parallel dynamic scheduling DoImportantStuff
% LocalWords:  lccp merged cell svg pdf
