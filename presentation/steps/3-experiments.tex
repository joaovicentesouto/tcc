\section{Experiments}

	% \pholder{Methodology}
	\begin{frame}[fragile]{Methodology}
		\begin{itemize}
			\item Evaluation of the performance of \textbf{data transfer services}
			\begin{itemize}
				\item Mailbox
				\item Portal
			\end{itemize}
			\item Micro-benchmarks stimulate the services with well-known \textbf{collective communication routines}
		\end{itemize}

		% Os experimentos realizados buscaram avaliar os serviços de transferência de dados, como mailbox e portal.
		% Entretanto, o sync foi utilizado para realizar a sincronização de todos os clusters envolvidos.
		% Para prover uma avaliação abrangente, os micro-benchmarks estimularam os serviços usando rotinas de comunicação coletivas.
	\end{frame}

	% \pholder{Micro-benchmarks}
	\begin{frame}[fragile]{Micro-benchmarks}
		\begin{overprint}
			\only<1>{
				\begin{itemize}
					\item \textbf{Broadcast}
					\item Gather
					\item AllGather
					\item Ping-Pong
				\end{itemize}
				\addfig[][width=.5\linewidth]{mpi-broadcast.pdf}
			}
			\only<2>{
				\begin{itemize}
					\item Broadcast
					\item \textbf{Gather}
					\item AllGather
					\item Ping-Pong
				\end{itemize}
				\addfig[][width=.5\linewidth]{mpi-gather.pdf}
			}
			\only<3>{
				\begin{itemize}
					\item Broadcast
					\item Gather
					\item \textbf{AllGather}
					\item Ping-Pong
				\end{itemize}
				\addfig[][width=.5\linewidth]{mpi-allgather.pdf}
			}
			\only<4>{
				\begin{itemize}
					\item Broadcast
					\item Gather
					\item AllGather
					\item \textbf{Ping-Pong}
				\end{itemize}
				\addfig[][width=.5\linewidth]{mpi-ping-pong.pdf}
			}
		\end{overprint}

		% Ao todo foram implementados 4 rotinas.
		% A rotina broadcast define um nó mestre enviando uma mesma mensagem para N escravos.
		% Opostamente, a rotina gather, define que todos os escravos enviem seus dados para o mestre.
		% A rotina AllGather é uma generalização do gather, no qual todos enviam e recebem dados de todos mundo.
		% Por fim, a rotina Ping-Pong representa as diversas requisições e respostas de um modelo cliente-servidor.
	\end{frame}

	% \pholder{Design}
	\begin{frame}[fragile]{Experimental Design}
		\begin{itemize}
			\item \textbf{Throughput of the Portal}
			\begin{itemize}
				\item Constant: 1 IO and 16 CCs
				\item Variable: 4, 8, 16, 32, 64 KB
			\end{itemize}
			\item \textbf{Latency of the Mailbox}
			\begin{itemize}
				\item Variable: 1 IO and 1 to 16 CCs
				\item Constant: 120 B
			\end{itemize}
			\item \textbf{50 iterations}, discarding first ten
			\item Standard error inferior to \textbf{1\%}
		\end{itemize}

		% Para o serviço do Portal, foi avaliado a taxa de transferência de dados envolvendo sempre um número fixo de cluster. O que variou a quantidade de dados, variando de 4 KB até 64 KB.
		% Já a latência da mailbox envolveu um número variável de clusters porque a mensagem trocada é fixada com o tamanho de 120 bytes.
		% Todos os experimentos tiveram 50 iterações, mas as 10 primeiras foram descartadas para ignorar variações provenientes da inicialização do experimento.
		% Por fim, todos os resultados obtiveram erro padrão menor que 1%.
	\end{frame}

	% \pholder{Portal Analysis}
	\begin{frame}[fragile]{Portal Analysis}
		\addfig[][width=.9\linewidth]{portal-throughput.pdf}

		% O gráfico mostra a taxa de transferência do Portal em MB/s relativo a quantidade de dados transferido.
		% Os resultados exibiram três comportamentos distintos.
		% O primeiro, o broadcast apresentou os piores resultados como experado, por causa do único emissor envolvido.
		% Segundo, Gather e Ping-Pong estão sobrepostos porque tiveram resultados similares.
		% Isso é causado por causa da serialização das leituras pelo mestre, o qual dita o fluxo dos dados.
		% Terceiro, o melhor dos resultados foi do AllGather, porque as comunicações ocorriam em pares de nós, o que permite o paralelismo das tranferências.
		% No contexto de Sistems Operacionais, onde teremos subsistemas requisitando grandes transferências de dados, como um sistema de paginação, podemos inferir que tamanhos entre 8 e 16 KB favorecem a taxa de transferência do Portal.
		% Apesar de tudo, acreditamos que ao eliminar as limitações da DMA, o Portal deve ter ganhos significativos.
	\end{frame}

	% \pholder{Mailbox Analysis}
	\begin{frame}[fragile]{Mailbox Analysis}
		\addfig[][width=.9\linewidth]{mailbox-latency.pdf}

		% Por outro lado, o gráfico da Mailbox apresenta a latencia em milisegundos relativo ao número de clusters de computação envolvidos.
		% Primeiramente, as rotinas Gather e AllGather apresentaram os melhores resultados por causa do  paralelismo do recebimento das N mensagens. A diferença do AllGather se encontra na necessidade de envio e leitura, diferente da Gather.
		% Segundo, o broadcast sofreu do mesmo mal da rotina no Portal, por causa do único emissor de mensagens.
		% Por último, o ping-pong apresentou os piores resultados, onde é possível notar que seus resultados são similares as somas dos resultados do gather e do broadcast.
		% O que indica que o ganho do mestre em receber as mensagens em paralelo foi atenuado pelo gargalo de enviar N mensagens.
		% Entretanto, de forma geral, os resultados mostraram que é possível suportar de forma eficiente essas rotinas no Nanvix Multikernel. Os resultados do Ping-pong, por sua vez, mostraram potenciais melhorias que devem ser estudadas na comunicação dos subsistemas do Nanvix.
	\end{frame}

% LocalWords:  template cls standalone GitHub Overleaf bugfixes SVGs
% LocalWords:  Re-empacotamento fontsize Makefile pdflatex imgs PDFs
% LocalWords:  shell-escape frames SVG brazil english lapesd-slides
% LocalWords:  disabletodonotes todonotes TODO's backup showbackup
% LocalWords:  hidebackup abntexcite abntex natbib nobib titleframe
% LocalWords:  frame showsections sidebar stopcountingframes default
% LocalWords:  thanksframe Thank You Questions referencesframe titulo
% LocalWords:  bibfiles pholder todonote placeholder inline addfig
% LocalWords:  opts graphicx addfiglw width Citations dijkstra Direct
% LocalWords:  Closure Parallel dynamic scheduling DoImportantStuff
% LocalWords:  lccp merged cell svg pdf
