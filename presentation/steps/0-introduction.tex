% Desse ponto em diante serão inseridos slides de pausa a cada \section
% \showsections
\section{Introduction}

	% Bom dia, primeiramente quero agradecer a banca pelo tempo gasto na avaliação do meu trabalho e agradecer a presença de todos.

	% Bom dia, agradeço a presença de todos.
	% Especialmente do meu orientador Professor Márcio Bastos Castro, do meu coorientador Pedro Henrique Penna e dos avaliadores, Rômulo Silva de Oliveira e Odorico Machado Mendizabal.

	% O meu trabalho de conclusão de curso envolve o desenvolvimento de mecanismos de comunicação para processadores lightweight manycores para o Sistema Operacional Nanvix

	% Conforme o esboço da apresentação, primeiro vou contextualizar e apresentar os objetivos do meu trabalho.
	% Em seguida, irei pontuar alguns conceitos básicos e apresentar o desenvolvimento.
	% Depois, irei mostrar os experimentos e discutir os resultados.
	% Por fim, apresentou as minhas conclusões.

	\begin{frame}{Processor Trend}
		\begin{overprint}
			\only<1>{\addfig[Historical evolution of transistor-based processors][width=.9\linewidth]{processor-trend-1.pdf}}
			\only<2>{\addfig[Historical evolution of transistor-based processors][width=.9\linewidth]{processor-trend-2.pdf}}
			\only<3>{\addfig[Historical evolution of transistor-based processors][width=.9\linewidth]{processor-trend-3.pdf}}
			\only<4>{\addfig[Historical evolution of transistor-based processors][width=.9\linewidth]{processor-trend-4.pdf}}
			\only<5>{\addfig[Historical evolution of transistor-based processors][width=.9\linewidth]{processor-trend-5.pdf}}
		\end{overprint}

		% Por muitos anos, o aumento da frequência dos processadores impulsionou os ganhos de desempenho dos sistemas computacionais modernos.
		% No entanto, entre 2000 e 2005, as altas temperaturas provocadas por essa técnica impuseram um limite físico, conhecido como barreira de frequência.
		% Paralelamente, o aprimoramento constante da tecnologia de semicondutores e o aperfeiçoamento de um único fluxo de execução amenizaram o impacto desse limite.
		% Essa barreira também influenciou diretamento o surgimento dos primeiros processadores com mais de 1 núcleo de processamento, chamados de multicores.
		% Em seguida, não demorou para que surgissem processadores com centenas a milhares de núcleos.
		% Conhecidos como manycores, eles apresentam memória distribuída e requerem uma rede-em-chip.
		% Atualmente, existe uma preocupação crescente com a relação entre a poder de processamento e o consumo de energia. Essa mudança de paradigma levou ao surgimento de uma nova classe de processadores paralelos, chamados Lightweight Manycores, que surgiram para fornecer alto paralelismo com baixo consumo de energia.
	\end{frame}

	\begin{frame}[fragile]{Lightweight Manycores Particularities}
		\begin{itemize}
			\item \textbf{Integrates thousands of low-power cores} grouped in clusters
			\item Copes with \textbf{MIMD workloads}
			\item \textbf{Relies on a Network-on-Chip (NoC)} % for fast and reliable message-passing communication
			\begin{itemize}
				\item Hybrid programming model
			\end{itemize}
			\item Has \textbf{constrained memory systems}
			\begin{itemize}
				\item Multiple and Small physical address spaces
				\item Missing cache-coherence
			\end{itemize}
			\item Features a \textbf{heterogeneous configuration}
		\end{itemize}

		% Lightweight Manycores diferem dos multicores e manycores tradicionais em vários pontos.
		% Eles integram centenas de núcleos de baixa potência agrupados em clusters;
		% Trabalham com cargas de trabalho com múltiplos fluxos de instrução e dados;
		% Dependem de uma rede-em-chip para comunicação entre cluster, o que força o uso de um modelo de programação híbrido, ou seja, memória compartilhada e troca de mensagens;
		% Apresentam sistemas de memória restritívos, com pequenos e multiplos espaços de endereçamento sem coerência de cache;
		% E possuem geralmente componentes heterogêneos.

	\end{frame}

	\begin{frame}[fragile]{Work Context}
		\begin{itemize}
			\item \textbf{Existing Challenges} % derives from existing \textbf{runtimes} and \textbf{Operating Systems}.
			\begin{itemize}
				\item Runtimes
				\item Operating Systems (OS)
			\end{itemize}
		\end{itemize}

		\begin{itemize}
			\item \textbf{OSs for Next-generation for lightweight manycores}
			\begin{itemize}
				\item Redesigned from scratch
				\item \textbf{Nanvix OS}
			\end{itemize}
		\end{itemize}

		% Parte dos desafios de se trabalhar com Lightweight Manycores originam-se de runtimes e sistemas operacionais existentes. Onde eles lidam parcialmente com as características do hardware ou escalabilidade do sistema.
		% Assim, nós acreditamos que os sistemas operacionais para a próxima geração de Lightweight Manycores devem ser reprojetados do básico levando em consideração suas restrições arquitetônicas.
		% Com base nessa idéia, buscamos desenvolver um sistema operacional distribuído baseado no modelo multikernel, chamado Nanvix OS.
	\end{frame}

\section{Goals}

	\begin{frame}[fragile]{Goals}
		\begin{itemize}
			\item Definition and proposal of an \textbf{Inter-Cluster Communication Interface} for lightweight manycores
		\end{itemize}

		\begin{itemize}
			\item Implementation in the \textbf{Nanvix HAL}
			\begin{itemize}
				\item Kalray MPPA-256 lightweight manycore processor
			\end{itemize}
		\end{itemize}

		\begin{itemize}
			\item Integration with the \textbf{Nanvix Microkernel}
		\end{itemize}

		\begin{itemize}
			\item Performance evaluation using synthetic micro-benchmarks
			\begin{itemize}
				\item \textbf{Collective communication routines}
			\end{itemize}
		\end{itemize}

		% Neste contexto, os objetivos do trabalho é definir e propor uma Interface de Comunicação entre Clusters para Lightweight Manycores.
		% Implementar essa interface na camada de abstração do Nanvix, focado no processodor MPPA-256.
		% Depois, integrar com o Nanvix Microkernel e avaliar a perfomance utilizando rotinas comunicação coletivas.
	\end{frame}

% LocalWords:  template cls standalone GitHub Overleaf bugfixes SVGs
% LocalWords:  Re-empacotamento fontsize Makefile pdflatex imgs PDFs
% LocalWords:  shell-escape frames SVG brazil english lapesd-slides
% LocalWords:  disabletodonotes todonotes TODO's backup showbackup
% LocalWords:  hidebackup abntexcite abntex natbib nobib titleframe
% LocalWords:  frame showsections sidebar stopcountingframes default
% LocalWords:  thanksframe Thank You Questions referencesframe titulo
% LocalWords:  bibfiles pholder todonote placeholder inline addfig
% LocalWords:  opts graphicx addfiglw width Citations dijkstra Direct
% LocalWords:  Closure Parallel dynamic scheduling DoImportantStuff
% LocalWords:  lccp merged cell svg pdf
