% Desse ponto em diante serão inseridos slides de pausa a cada \section
% \showsections
\section{Introduction}

	% Bom dia, primeiramente quero agradecer a banca pelo tempo gasto na avaliação do meu trabalho e agradecer a presença de todos.

	% Bom dia, agradeço a presença de todos.
	% Especialmente do meu orientador Professor Márcio Bastos Castro, do meu coorientador Pedro Henrique Penna e dos avaliadores, Rômulo Silva de Oliveira e Odorico Machado Mendizabal.

	% O meu trabalho de conclusão de curso envolve o desenvolvimento de mecanismos de comunicação para processadores lightweight manycores para o Sistema Operacional Nanvix

	% Conforme o esboço da apresentação, primeiro vou contextualizar e apresentar os objetivos do meu trabalho.
	% Em seguida, irei pontuar alguns conceitos básicos e apresentar o desenvolvimento.
	% Depois, irei mostrar os experimentos e discutir os resultados.
	% Por fim, apresentou as minhas conclusões.

	\begin{frame}{Historical Evolution and Trend of Processors}
		\begin{overprint}
			\only<1>{
				\addfig[][width=.9\linewidth]{processor-trend-1.pdf}
				\begin{center}
					Processors \textbf{performance improvement}
				\end{center}
			}
			\only<2>{
				\addfig[][width=.9\linewidth]{processor-trend-2.pdf}
				\begin{center}
					Limited by \textbf{Frequency Barrier}
				\end{center}
			}
			\only<3>{
				\addfig[][width=.9\linewidth]{processor-trend-3.pdf}
				\begin{center}
					Softened by \textbf{technological advancement}
				\end{center}
			}
			\only<4>{
				\addfig[][width=.9\linewidth]{processor-trend-4.pdf}
				\begin{center}
					Emergence of \textbf{Multicores and Manycores}
				\end{center}
			}
			\only<5>{
				\addfig[][width=.9\linewidth]{processor-trend-5.pdf}
				\begin{center}
					% \textbf{Processing power versus power consumption\\led to the advent of Lightweight Manycores}
					Advent of \textbf{Lightweight Manycores}
				\end{center}
			}
		\end{overprint}

		% Por muitos anos, o aumento de frequência impulsionou o ganho de desempenho dos sistemas computacionais.
		% Entretanto, em meados de 2005, a barreira de frequência causada pelas altas temperaturas dessa técnica limitou os processadores.
		% Paralelamente, os avanços tecnológicos dos semicondutores e componentes internos amenizaram o impacto dessa limitação.
		% A barreira de frequência também levou ao surgimento dos processadores com dezenas à milhares núcleos físicos, os chamadas multicores e manycores.
		% Contudo, a crescente preocupação com a relação entre poder de processamento e consumo de energia levaram ao surgimento dos Lightweight Manycores.
	\end{frame}

	\begin{frame}[fragile, t]{Lightweight Manycores Particularities}
			\only<1>{
				\addfig[\small{Overview of a Manycore}][width=.75\linewidth]{lightweight-manycore.pdf}

				\vspace{-0.25cm}

				\begin{itemize}
					\item \textbf{Hundreds of Lightweight Cores}
					\begin{itemize}
						\item Expose Massive thread-level parallelism
						\item Feature low-power consuption
						\item Target MIMD workloads
					\end{itemize}
					\item Distributed Memory Architecture
					\item On-Chip Heterogeneity
				\end{itemize}
			}
			\only<2>{
				\addfig[\small{Overview of a Manycore}][width=.75\linewidth]{lightweight-manycore.pdf}

				\vspace{-0.25cm}

				\begin{itemize}
					\item Hundreds of Lightweight Cores
					\item \textbf{Distributed Memory Architecture}
					\begin{itemize}
						\item Grants scalability
						\item Relies on a Network-on-Chip (NoC)
						\item Has constrained memory systems
					\end{itemize}
					\item On-Chip Heterogeneity
				\end{itemize}
			}
			\only<3>{
				\addfig[\small{Overview of a Manycore}][width=.75\linewidth]{lightweight-manycore.pdf}

				\vspace{-0.25cm}

				\begin{itemize}
					\item Hundreds of Lightweight Cores
					\item Distributed Memory Architecture
					\item \textbf{On-Chip Heterogeneity}
					\begin{itemize}
						\item Features different components
					\end{itemize}
				\end{itemize}
			}

		% \begin{columns}[totalwidth=\linewidth,t]
		% 	\column{0.55\linewidth}

		% 		% \begin{itemize}
		% 		% 	\item \textbf{Integrates thousands of low-power cores} grouped in clusters
		% 		% 	\item Copes with \textbf{MIMD workloads}
		% 		% 	\item \textbf{Relies on a Network-on-Chip (NoC)} % for fast and reliable message-passing communication
		% 		% 	\begin{itemize}
		% 		% 		\item Hybrid programming model
		% 		% 	\end{itemize}
		% 		% 	\item Has \textbf{constrained memory systems}
		% 		% 	\begin{itemize}
		% 		% 		\item Multiple and Small physical address spaces
		% 		% 		\item Missing cache-coherence
		% 		% 	\end{itemize}
		% 		% 	\item Features a \textbf{heterogeneous configuration}
		% 		% \end{itemize}
				
		% 			\begin{itemize}
		% 				\item \textbf{Hundreds of Lightweight Cores}
		% 				\begin{itemize}
		% 					\item Expose Massive thread-level parallelism
		% 					\item Feature low-power consuption
		% 					\item Target MIMD workloads
		% 				\end{itemize}
		% 				\item \textbf{Distributed Memory Architecture}
		% 				\begin{itemize}
		% 					\item Grants scalability
		% 					\item Has constrained memory systems
		% 					\item Requires hybrid programming
		% 				\end{itemize}
		% 				\item \textbf{On-Chip Heterogeneity}
		% 				\begin{itemize}
		% 					\item Features different components
		% 				\end{itemize}
		% 			\end{itemize}

		% 		% \only<1>{
		% 		% 	\begin{itemize}
		% 		% 		\item \textbf{Hundreds of Lightweight Cores}
		% 		% 		\begin{itemize}
		% 		% 			\item Expose Massive thread-level parallelism
		% 		% 			\item Feature low-power consuption
		% 		% 			\item Target MIMD workloads
		% 		% 		\end{itemize}
		% 		% 		\item Distributed Memory Architecture
		% 		% 		\item On-Chip Heterogeneity
		% 		% 	\end{itemize}
		% 		% }
		% 		% \only<2>{
		% 		% 	\begin{itemize}
		% 		% 		\item Hundreds of Lightweight Cores
		% 		% 		\item \textbf{Distributed Memory Architecture}
		% 		% 		\begin{itemize}
		% 		% 			\item Grants scalability
		% 		% 			\item Has constrained memory systems
		% 		% 			\item Requires hybrid programming
		% 		% 		\end{itemize}
		% 		% 		\item On-Chip Heterogeneity
		% 		% 	\end{itemize}
		% 		% }
		% 		% \only<3>{
		% 		% 	\begin{itemize}
		% 		% 		\item Hundreds of Lightweight Cores
		% 		% 		\item Distributed Memory Architecture
		% 		% 		\item \textbf{On-Chip Heterogeneity}
		% 		% 		\begin{itemize}
		% 		% 			\item Features different components
		% 		% 		\end{itemize}
		% 		% 	\end{itemize}
		% 		% }

		% 	\column{0.45\linewidth}
		% 		\addfig[Overview of a manycore][width=\linewidth]{lightweight-manycore.pdf}
		% \end{columns}

		% Lightweight Manycores diferem dos multicores e manycores tradicionais em vários pontos.
		% Eles integram centenas de núcleos de baixa potência agrupados em clusters;
		% Trabalham com cargas de trabalho com múltiplos fluxos de instrução e dados;
		% Dependem de uma rede-em-chip para comunicação entre cluster, o que força o uso de um modelo de programação híbrido, ou seja, memória compartilhada e troca de mensagens;
		% Apresentam sistemas de memória restritívos, com pequenos e multiplos espaços de endereçamento sem coerência de cache;
		% E possuem geralmente componentes heterogêneos.

	\end{frame}

	\begin{frame}[fragile]{Context and Goals}
		\begin{itemize}
			\item \textbf{Work context}
			\begin{itemize}
				\item Challenges arising from \textbf{runtimes} and \textbf{OSs}
				\item \textbf{OSs for Next-generation for lightweight manycores}
				\item \textbf{Nanvix OS}
			\end{itemize}
		\end{itemize}

		\begin{itemize}
			\item \textbf{Goals}
			\begin{itemize}
				\item Definition and proposal of an \textbf{Inter-Cluster Communication Interface} for lightweight manycores
				\item Implementation in the \textbf{Nanvix HAL} on Kalray MPPA-256 Lightweight Manycore Processor
				\item Integration with the \textbf{Nanvix Microkernel}
				\item Performance evaluation using synthetic micro-benchmarks with \textbf{Collective communication routines}
			\end{itemize}
		\end{itemize}

		% Parte dos desafios de se trabalhar com Lightweight Manycores originam-se de runtimes e sistemas operacionais existentes. Onde eles lidam parcialmente com as características do hardware ou escalabilidade do sistema.
		% Assim, nós acreditamos que os sistemas operacionais para a próxima geração de Lightweight Manycores devem ser reprojetados do básico levando em consideração suas restrições arquitetônicas.
		% Com base nessa idéia, buscamos desenvolver um sistema operacional distribuído baseado no modelo multikernel, chamado Nanvix OS.
	\end{frame}

	% \begin{frame}[fragile]{Goals}
		

	% 	% Neste contexto, os objetivos do trabalho é definir e propor uma Interface de Comunicação entre Clusters para Lightweight Manycores.
	% 	% Implementar essa interface na camada de abstração do Nanvix, focado no processodor MPPA-256.
	% 	% Depois, integrar com o Nanvix Microkernel e avaliar a perfomance utilizando rotinas comunicação coletivas.
	% \end{frame}

% LocalWords:  template cls standalone GitHub Overleaf bugfixes SVGs
% LocalWords:  Re-empacotamento fontsize Makefile pdflatex imgs PDFs
% LocalWords:  shell-escape frames SVG brazil english lapesd-slides
% LocalWords:  disabletodonotes todonotes TODO's backup showbackup
% LocalWords:  hidebackup abntexcite abntex natbib nobib titleframe
% LocalWords:  frame showsections sidebar stopcountingframes default
% LocalWords:  thanksframe Thank You Questions referencesframe titulo
% LocalWords:  bibfiles pholder todonote placeholder inline addfig
% LocalWords:  opts graphicx addfiglw width Citations dijkstra Direct
% LocalWords:  Closure Parallel dynamic scheduling DoImportantStuff
% LocalWords:  lccp merged cell svg pdf
