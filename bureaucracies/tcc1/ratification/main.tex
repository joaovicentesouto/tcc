\documentclass[12pt]{article}
\usepackage[a4paper,margin=2cm]{geometry}

\usepackage[T1]{fontenc}
\usepackage[utf8]{inputenc}

\usepackage{mathptmx}
\usepackage{tabularx}
\usepackage{multirow}
\usepackage{makecell}
\usepackage{pgfgantt}
\usepackage{graphics}
\usepackage{graphicx}
\begin{document}

\pagestyle{empty}

\begin{centering}

	\textbf{DEPARTAMENTO DE INFORMÁTICA E ESTATÍSTICA -- CTC -- UFSC}

	\textbf{RATIFICAÇÃO DE PLANO DE TRABALHO DO SEMESTRE \\ PARA DESENVOLVIMENTO DE TCC}

\end{centering}


\vspace{1em}
\setlength\extrarowheight{5pt}
\begin{tabular}{l l l}
	\textbf{Disciplina:} & ( \textbf{X} ) TCC 1 &  ( \textbf{} ) TCC 2\\
	\textbf{Curso:}      & ( \textbf{X} ) CCO   & ( ) SIN             \\
	\textbf{Autor:}      & João Vicente Souto   &                     \\
\end{tabular}
\vspace{0.5cm}
\\
\begin{tabular}{l l}
\vspace{0.2cm}
	\textbf{Título:} & \makecell{Desenvolvimento de Componentes e Serviços de um Microkernel
	                                    \\para o Processador Manycore MPPA-256}\\
	\textbf{Professor responsável:} & Prof. Dr. Márcio Bastos Castro\\
\end{tabular}

\vspace{2em}
{\large \textbf{Objetivos}}
\\

\textbf{Objetivo geral:}
O objetivo do TCC é pesquisar e desenvolver componentes e serviços de um
microkernel para o processador \textit{manycore} MPPA-256. Esses elementos
buscarão fornecer a melhor relação entre desempenho, escalabilidade e 
portabilidade, reduzindo assim as dificuldades encontradas no desenvolvimento
de aplicações para o processador.

\vspace{2em}
{\large \textbf{Cronograma}}
\\

%%%% Modelo com X's %%%%

\begin{tabularx}{\linewidth}{|X|*{6}{c|}}
	\hline
	\multicolumn{1}{|c|}{\multirow{2}{*}{Etapas}} & \multicolumn{6}{|c|}{Meses}\\ \cline{2-7}
	& jan & fev & mar & abr & maio & jun \\ \hline

	Estudar conceitos de Sistemas Operacionais
	&  x  &  x  &     &     &     &     \\ \hline

	Estudar Componentes e Serviços existentes
	&     &  x  &  x  &     &     &     \\ \hline

	Projetar os Componentes e Serviços para o microkernel
	&     &     &  x  &  x  &  x  &  x  \\ \hline

	Desenvolver implementação
	&     &     &     &  x  &  x  &  x  \\ \hline

	Testar e validar a implementação
	&     &     &     &     &  x  &  x  \\ \hline

	Documentar o aprendizado
	&  x  &  x  &  x  &  x  &  x  &  x  \\ \hline

\end{tabularx}

%%%% Modelo com barras %%%%

% \begin{figure}[!h]
	% \begin{center}

	% 	\begin{ganttchart}[
	% 		x unit=1.6cm,
	% 		y unit title=1cm,
	% 		y unit chart=1cm,
	% 		hgrid,
	% 		vgrid={{dotted, dotted, dotted, dotted, dotted, dotted}},
	% 		% title label font=\3scriptsize,
	% 		title/.append style={fill=gray!30},
	% 		title height=1,
	% 		bar/.append style={fill=gray!30,rounded corners=2pt},
	% 		bar label font=\scriptsize,
	% 		group label font=\scriptsize,
	% 	]{1}{6}

	% 	\gantttitle{\textbf{Meses}}{6} \\
	% 	\gantttitle{\textbf{2019}}{6} \\
	% 	\gantttitlelist{1,2,3,4,5,6}{1} \\
	% 	\ganttbar{1. Estudar abstrações IPC}{1}{3} \\
	% 	\ganttbar{2. Desenvolver as abstrações}{2}{4} \\
	% 	\ganttbar{2. Desenvolver os experimentos}{4}{5} \\
	% 	\ganttbar{3. Testar e validar as abstrações}{5}{5} \\
	% 	\ganttbar{3. Corrigir e aperfeiçoar a implementação}{5}{6} \\
	% 	\ganttbar{4. Documentar o aprendido}{1}{6} \\

	% 	\end{ganttchart}
	% \end{center}
	% \label{tab:cronograma}
% \end{figure}

\vspace{3em}

\begin{centering}

	\fbox{\begin{minipage}[c][6em][c]{0.7\textwidth}
		{\center \textbf{Preenchimento pelo Professor responsável pelo TCC}\\[1em]}

		\qquad $(\quad)$ \ Ciente e de acordo.

		\qquad Data: \_\_ / \_\_ / \_\_
	\end{minipage}}

\end{centering}
\end{document}
