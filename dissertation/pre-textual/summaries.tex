% Assume-se que \pretextual já foi feito

\begin{resumo}[Resumo]
    Em conjunto com a maior escalabilidade de desempenho e eficiência energética,
    os \textit{lightweight manycores} trouxeram um novo conjunto de desafios no
    desenvolvimento de software, provenientes de suas particularidades arquitetônicas.
    Nesse contexto, os sistemas operacionais são essenciais porque permitem abstrair
    as características do hardware sob uma perspectiva simplificada e eficaz.
    Assim, os sistemas operacionais tornam o desenvolvimento de aplicações
    menos oneroso e mais eficiente.
    No entanto, parte dos desafios encontrados em \textit{lightweight manycores}
    deriva dos runtimes e dos sistemas operacionais existentes que não lidam
    completamente com as complexidades desses processadores.
    Assim, acreditamos que os sistemas operacionais para a próxima geração de
    \textit{lightweight manycores} devem ser reprojetados do zero para lidar
    com suas severas restrições arquitetônicas.
    Em particular, devido à natureza distribuída dos \textit{manycores}, as
    abstrações de comunicação desempenham um papel crucial na escalabilidade
    e desempenho das aplicações.
    Neste cenário, o objetivo deste trabalho é desenvolver um módulo de comunicação
    entre clusters para o processador \textit{manycore} emergente MPPA-256.
    Este módulo faz parte de um Camada de Abstração de Hardwarre genérico e
    flexível para \textit{lightweight manycores} que lida com os principais
    problemas encontrados no projeto de um sistema operacional para esses processadores.
    Encima desse módulo, serviços de comunicação também serão propostos para
    um sistema operacional baseado no modelo \textit{microkernel}, que busca
    fornecer um esqueleto básico para as abstrações de sistema.
    Essas contribuições estão inseridas em um contexto de pesquisa mais amplo
    que procura investigar um sistema operacional distribuído completo baseado
    em uma abordagem multikernel para esses processadores.

    % Atenção! a BU exige separação através de ponto (.). Ela recomanda de 3 a 5 keywords
    \vspace{\baselineskip} 
    \textbf{Palavras-chave:} Palavra-chave. Ponto como separador. Bla.
\end{resumo}


\begin{resumo}[Resumo Estendido]
%%%%%%%%%%%%%%%%%%%%%%%%%%%%%%%%%%%%%%%%%%%%%%%%%%%%%%%%%%%%%%%%%%%%%%
% Atenção: normas e templates contraditórios!!!                    %%%
%%%%%%%%%%%%%%%%%%%%%%%%%%%%%%%%%%%%%%%%%%%%%%%%%%%%%%%%%%%%%%%%%%%%%%
% - Modelo da BU: https://repositorio.ufsc.br/handle/123456789/197458
% - A BU exige no **mínimo** 2 páginas e no **máximo** 5
% - Regimento do PPGCC, Art 40 Entende-se  por  resumo  estendido  um  documento  que  contenha  as  informações  mais  relevantes  de  cada  capítulo  da  tese  ou  da  dissertação.
% O mais seguro é ignorar o regimento e seguir a BU.
    % Atenção! A BU diz que o resumo **deve** conter as seções abaixo!
  \section*{Introdução} % Deve ser  subsection*, devido a formatação usada no modelo
  A hifenização é alterada para \texttt{brazil}, mesmo para documentos em inglês. Descrever brevemente esses itens exigidos pela BU. Como a RN 95/CUn/2017 é mais recente e impõe outras regras a revelia de regimentos e regulamentos, é mais sábio obedecê-la. Lembre que esse resumo estendido deve term entre 2 e 5 páginas.
  
  \lipsum[1]
  \section*{Objetivos} 
  \lipsum[21]
  \section*{Metodologia} 
  \lipsum[3]
  \section*{Resultados e Discussão} 
  \lipsum[4]
  \section*{Considerações Finais} 
  \lipsum[5]

  \vspace{\baselineskip}  % Atenção! manter igual ao resumo
  \textbf{Palavras-chave:} Palavra-chave. Outra Palavra-chave composta. Bla.
\end{resumo}

\begin{abstract}
    Jointly with further performance scalability and energy efficiency,
    lightweight manycores brought a new set of challenges in software
    development coming from their architectural particularities.
    In this context, \oss are essential because they allow to abstract
    the characteristics of the hardware under a simplified and productive
    perspective.
    Thus, \oss make application development less costly and more efficient.
    However, part of the challenges encountered in lightweight manycores
    derives from existing runtimes and \oss that do not completely handle
    extant intricacies.
    So, we believe that \oss for the next-generation of lightweight
    manycores must be redesigned from scratch to cope with their
    tight architectural constraints.
    In particular, due to the distributed nature of the manycores,
    communication abstractions play a crucial role in the scalability
    and performance of these processors.
    In this scenario, the goal of this work is to develop an inter-cluster
    communication module for the emergent \mppa Lightweight Manycore Processor.
    This module is part of a generic and flexible \hal for lightweight
    manycores that cope with the key issues encountered in designing
    an \os for these processors.
    On top of the module, communication services will also be proposed
    for a microkernel-based \os that seeks to provide bare bones for
    system abstractions.
    These contributions are embedded in a broader research context
    that seeks to investigate a fully-featured distributed \os based
    on a multikernel approach to these processors.

    \vspace{\baselineskip} 
    \textbf{Keywords:} Keyword. Another Compound Keyword. Bla.
\end{abstract}

%%% Local Variables:
%%% mode: latex
%%% TeX-master: "main"
%%% End:
