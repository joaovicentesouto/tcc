% Assume-se que \pretextual já foi feito

\begin{resumo}[Resumo]
	Em conjunto com a maior escalabilidade e eficiência energética, os
	\textit{\lightweight \manycores} trouxeram um novo conjunto de desafios no
	desenvolvimento de software provenientes de suas particularidades
	arquiteturais. Neste contexto, sistemas operacionais tornam o
	desenvolvimento de aplicações menos onerosos, menos suscetíveis a
	erros e mais eficientes. A camada de abstração provido pelos
	sistemas operacionais suprime as características do hardware sob
	uma perspectiva simplificada e eficaz. No entanto, parte dos desafios
	de desenvolvimento encontrados em \textit{\lightweight \manycores} deriva
	diretamente de \textit{runtimes} e sistemas operacionais existentes,
	que não lidam completamente com a complexidade arquitetural desses
	processadores. Acreditamos que sistemas operacionais para a
	próxima geração de \textit{\lightweight \manycores} necessitam ser repensados
	a partir de seus conceitos básicos considerando as severas
	restrições arquiteturais. Em particular, as abstrações de comunicação
	desempenham um papel crucial na escalabilidade e desempenho das
	aplicações devido à natureza distribuída dos \textit{\manycores}. O objetivo
	deste trabalho é propor mecanismos de comunicação entre clusters
	para o processador \textit{\manycore} emergente MPPA-256. Estes mecanismos
	fazem parte de uma Camada de Abstração de Hardware (HAL) genérica
	e flexível para \textit{\lightweight \manycores} que lida
	diretamente com os principais problemas encontrados no projeto de um
	sistema operacional para esses processadores. Sob estes mecanismos,
	serviços de comunicação também serão propostos para um sistema
	operacional baseado no modelo \textit{microkernel}, que busca fornecer
	um esqueleto básico para as abstrações de comunicação. As contribuições
	desta dissertação estão inseridas em um contexto de pesquisa mais
	amplo, que procura investigar a criação de um sistema operacional
	distribuído baseado em uma abordagem multikernel, denominado
	\textit{Nanvix OS}. O Nanvix OS se concentrará em questões de
	programabilidade e portabilidade através de um sistema operacional
	compatível com o padrão POSIX para \textit{\lightweight \manycore}. Os
	resultados mostram como algoritmos distribuídos conhecidos podem
	ser eficientemente suportados pelo Nanvix OS e incentivam melhorias
	providas pelo uso adequado dos aceleradores DMA.

	% Atenção! a BU exige separação através de ponto (.). Ela recomanda de 3 a 5 keywords
	\vspace{\baselineskip}
	\textbf{Palavras-chave:} HAL. Sistema Operacional Distribuído. \textit{Lightweight Manycore}. Kalray MPPA-256.
\end{resumo}

\begin{abstract}
	Jointly with further scalability and energy efficiency, \lightweight
	\manycores brought a new set of challenges in software development
	coming from their architectural particularities. In this context,
	\oss make application development less costly, less error-prone,
	and more efficient. The abstraction layer provided by \oss suppresses
	hardware characteristics from a simplified and productive perspective.
	However, part of the development challenges encountered in lightweight
	manycores stems from existing runtimes and \oss, which do not entirely
	address the complexity of these processors. We believe that \oss for the
	next generation of lightweight manycores must be redesigned from scratch
	to cope with their tight architectural constraints. In particular,
	communication abstractions play a crucial role in application scalability
	and performance due to the distributed nature of manycores. The purpose
	of this undergraduate dissertation is to propose an inter-cluster
	communication facility for the emerging manycore MPPA-256 processor.
	This facility is part of a generic and flexible \hal that deals directly
	with the key issues encountered in designing an \os for these processors.
	Above this facility, communication services will also be proposed for an
	\os based on the \textit{microkernel} model, which seeks to provide a
	basic framework for communication abstractions. The contributions of this
	undergraduate dissertation are embedded in a broader research context
	that aims to investigate the creation of a distributed \os based on a
	multikernel approach, called \textit{\nanvixos}. \nanvixos focuses on
	programmability and portability issues for manycores through a
	POSIX-compliant \os. The results present how well known
    distributed algorithms can be efficiently supported by \nanvixos and
    encourage improvements provided by the proper use of \dma accelerators.

	\vspace{\baselineskip}
	\textbf{Keywords:} HAL. Distributed Operating System. Lightweight Manycore. Kalray MPPA-256.
\end{abstract}

%%% Local Variables:
%%% mode: latex
%%% TeX-master: "main"
%%% End:
