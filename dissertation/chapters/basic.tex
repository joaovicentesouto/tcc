\chapter{Basic Concepts}
\label{ch.basic}

\section{Figure Examples}

Figure.

\begin{figure}[htb]
    \includegraphics[width=.4\textwidth]{images/ufsc_pb.pdf}
    
    \caption[Short Caption]{
        Long version of caption.
    }
\label{fig.cu}
\end{figure}

\subsection{Sub Figures}
Ref global Figure~\ref{fig.logo} and its subfigures~\ref{sfig.logoA}, \ref{sfig.logoB} and \ref{sfig.logoC}.

\begin{figure}[th]
    \subfloat[]{ \includegraphics[width=.2\linewidth]{images/ufsc_pb.pdf} \label{sfig.logoA} }
    \subfloat[]{ \includegraphics[width=.2\linewidth]{images/ufsc_pb.pdf} \label{sfig.logoB} }
    
    \subfloat[]{ \includegraphics[width=.2\linewidth]{images/ufsc_pb.pdf} \label{sfig.logoC} }
    
    \caption[Short Caption 2]{
        Long version of caption 2.
    }
\label{fig.logo}
\end{figure}

\section{Citations}
\citeonline{Wiegand03} made a nice job.
Citation \cite{Wiegand03}.

\section{Acronym}
See more in wiki\footnote{https://en.wikibooks.org/wiki/LaTeX/Glossary}.
\begin{enumerate}
    \item \gls{dag}.
    \item \glspl{dag}.
    \item \glspl{sram}.
    \item \gls{sram}.
    \item \acrfull{dag}
    \item \acrfullpl{dag}
    \item \test
    \item \test
    \item \tests
\end{enumerate}

\section{Table}

\begin{table}[h]
    \caption[Short Caption.]{
        Table caption here.
    }

    \resizebox{\linewidth}{!}{
        \begin{tabular}{@{}|c|c|c|c|c|c|c|c|c|c|@{}}
            \hline
              & \textbf{June}      & \textbf{July}      & \textbf{August}       & \textbf{September}     & \textbf{October}         & \textbf{November}         & \textbf{December}          & \textbf{January}         & \textbf{February}         \\ \hline
            \textbf{P1} & X & X  &   &   &   &   &   &   &   \\ \hline
            \textbf{P2} &   & X & X &   &   &   &   &   &   \\ \hline
            \textbf{P3} &   &  &  & X &   &   &   &   &   \\ \hline
            \textbf{P4} &   &  &  & X & X  & X  &   &   &   \\ \hline
            \textbf{P5} &   &   &   &   & X & X & X  &   &   \\ \hline
            \textbf{P6} &  &  &  &  &  &  & X & X  &   \\ \hline
            \textbf{P7} &   &   &   &   &   &   &   & X & X \\ \hline
            \textbf{P8} &   &   &   &   &   &   &   &   & X \\ \hline
        \end{tabular}
    }

\label{tab.table}
\end{table}

\section{Equations}
You can use symbols (\primes) from /init/math.

\begin{equation}
    % \displaystyle 
    \jcost(A, B) = + \jlambda \times A \in \integers \in \primes
\label{eq.test}
\end{equation}


\section{Algorithm}

\begin{algorithm}
    \DontPrintSemicolon % Some LaTeX compilers require you to use \dontprintsemicolon instead
    \KwIn{$A=[a_1, a_2, \ldots, a_n]$}
    \KwOut{Max value}
    $max \gets a_1$\;
    \For{$i \gets 2$ \textbf{to} $n$} {
      \If{$a_i > max$} {
        $max \gets a_i$\;
      }
    }
    \Return{$max$}\;
    
    \caption[Short]{
        Max finds the maximum number
    }
    \label{alg.max}
\end{algorithm}

\begin{algorithm}
    \DontPrintSemicolon % Some LaTeX compilers require you to use \dontprintsemicolon instead
    \KwIn{$A=[a_1, a_2, \ldots, a_n]$}
    \KwOut{Max value}
    $max \gets a_1$\;
    \For{$i \gets 2$ \textbf{to} $n$} {
      \If{$a_i > max$} {
        $max \gets a_i$\;
      }
    }
    \Return{$max$}\;
    
    \caption[Short]{
        Max finds the maximum number
    }
    \label{alg.max2}
\end{algorithm}