\chapter{Related Work}
\label{ch.related-work}

The proposal of this work is related to several other research work.
First, some research papers describing state-of-the-art \textit{lightweight} \manycores
processors will be cited. Further, research on different \oss
proposed for such processors will be highlighted.

\section{Lightweight Manycore Processors}
\label{sec.works.manycores}

	In addition to \mppa, many research studies exemplify the wide variety of
	architectural possibilities of \textit{lightweight} \manycores.
	For instance, Olofsson \etal~\cite{olofsson2014} introduce \epiphany as a
	high-performance energy-efficient \manycore architecture suitable for
	real-time embedded systems.
	The architecture consists of nodes communicating through three 2D mesh \nocs
	with a distributed shared-memory model without coherence protocol.
	Each node has one \risc \cpu, multi-banked local memory, a \dma engine,
	an event monitor and a network interface.
	The three \nocs, named rMesh, cMesh, and xMesh, are independent and scalable.
	They implement a packet-switched model with four duplex links at every node.
	When a node wants to read a remote data, first, it sends a read request over
	the rMesh and waits for a signal to arrive on cMesh.
		\todo{"On the another hand" does not make sense.}
	On the another hand, the write operation allows the node to continue running
	while data is tranfered through over xMesh.

		\todo{"On the another hand" does not make sense AGAIN.}
	On the other hand, for help and facilitate on the \manycore processor design,
	Wallentowitz \etal~\cite{Wallentowitz2013} presents the open-source \optimsoc
	framework, which allows build a
		\manycore \soc \todo[inline]{Maybe this is not correct, reread the paper.}
	and simulate it on a computer or
	synthesize it on an \fpga.
	The \pe are \openrisc~\footnote{https://opencores.org/openrisc}
		\todo[inline]{The number is with extra space.}
	processor organized in tiles.
	The central architectural element is the LISNoC, a \textit{packet-switched \noc}
	that implements virtual channels to avoid message-dependent deadlocks.
	The LISNoC support various network topologies, depending only on the tiles organization.
	Precisely, a \textit{network adapter} handles the memory transfers between
	tile and the memory and provides hardware means to a message-passing communication
	model among tiles.
	The tiles organization and the network topology also allows handling communication
	in three different ways.
	First, when using distributed memory, communication is performed through the
	message-exchange between the tiles.
	Second, by using a partitioned global address space scattered through the
	local memories of the tiles, the network adapter can perform \mpu-like memory
	address translation hiding the exchange of messages without providing cache coherence.
	Finally, Wallentowitz \etal study policy of write-through snooping to provide
	a global memory shared among tiles with cache coherence.

	Similarly, Kurth \etal~\cite{Kurth2017} introduce the \hero, which unites an \arm
	host processor with a fully modifiable \riscv \manycore implemented on an \fpga.
	The \pmca uses a multi-cluster design and also relies on multi-banked memory, called \spm.
	Multi-channel \dma engines had substituted the data caches, transferring data between
	a shared L1 \spm and all remote \spms or a shared main memory.
	Communication to main memory passes through software-managed lightweight \rab.
	The \rab performs the translation of the virtual-to-physical address, likes an \mmu.
	In this way, clusters can share virtual address pointers.
	Besides, exists different designs for the shared instruction caches and
	top-level interconnection such as bus or \noc.

\section{Operating Systems for Lightweight Manycores}
\label{sec.works.os}

	Baumann \etal~\cite{Baumann2009} proposed a new \os architecture for scalable multicore
	systems, called \multikernel.
	In their perspective, the future of the \oss is on embracing the networked nature
	of the machines based on distributed systems ideas.
	Assuming the cores are independent nodes of a network, they build the tradicional
	\os functionalities as fully-featured processes.
	These processes communicate via message-passing and do not share the internal
	structures of the \os.
	The work showed how expensive it is to maintain a state of the \os through
	shared-memory instead of exchanging messages and the subsequent increase of
	the complexity of cache-coherence protocols.
	The \multikernel implementation, named Barelfish, follows three design principles.
	First, \textit{Make all inter-core communication explicit} turns the system
	amenable to human or automated analysis because processes communicate only
	through well-defined interfaces.
	Second, \textit{Make \os structures hardware-neutral} makes the hardware-independent
	code easy to debug, optimize, and facilitates the deployment of \os for new
	processor types, avoiding rework.
	And lastly, \textit{View \os state as replicated instead of shared} improves system
	scalability.

	In Wisniewski~\cite{Wisniewski2014} \etal, the concept of scalability was pushed
	to the extreme, thinking on \hpc.
	The principal motivation is the creation of an \os that simultaneously supports
	programmability, through support \linux \api, and provides a lightweight kernel
	to performance, scalability, and reliability.
	The \os, named \mos, provides as much of the hardware resources as
	possible to the \hpc applications.
		\todo{"On the another hand" does not make sense AGAIN.}
	On the other hand, the Linux kernel
	component acts as a service that provides Linux functionalities.

	Similarly, Kluge \etal~\cite{kluge2014} developed the \moosca.
	With \moosca, they introduce abstractions that are easily composed, called Nodes,
	Channels and, Servers.
	Where Nodes represent execution resources, Channels represent communication
	resources, \eg \noc resources, and lastly, Servers are nodes that provide
	services to client Nodes.
	To meet safety-critical requirements, they partition \manycore and distribute
	replicas of Servers, turning the whole system more predictable.
	However, in order to deal with interferences in shared resources,
	usage policies are introduced to make possible the prediction of system behavior.

	Finally, Nightingale \etal~\cite{nightingale2009} presents the Helios \os to
	simplify the process of writing, deploy, and optimize an application across
	heterogeneous cores.
	They use the microkernel model, naming \textit{satellite kernel}, to export
	a uniform and straightforward set of \os abstractions.
	The most important design decisions were to avoid unnecessary remote communication
	by thinking about the penalty they have in \numa domains.
	Also, request the minimum of hardware primitives so that architectures with many
	constraints can be ported.
	Moreover, request the minimum hardware resources to support architectures with little
	computational power or memory constraints.

\section{Discussion}

	\autoref{sec.works.manycores} exemplifies how \manycore architectures can be
	grouped over a common logic perspective.
	They all have one or more logical units distributed and incorporated on clusters.
	The clusters, interconnected through a network, communicate by message-exchange.
	However, due to the domain for which these processors were designed, they end up
	presenting several differences among them at the hardware level.

		\todo{"On the another hand" does not make sense AGAIN.}
	On the other hand, \autoref{sec.works.os} presents studies of \oss that focus on
	mitigating these differences in order to provide greater programmability and portability.
	Nevertheless, they span the entire development spectrum, delivering to end-users
	high-level abstractions for increase productivity.
	In this way, it is possible to notice the extensive rework that all of them had in
	dealing with the closest layer of hardware that, as observed, can be abstracted to
	a common perspective.

	In this context, the proposed \hal deal with the lowest level possible with a focus
	on the aspects that make it challenging to work with \manycores.
	Thus, different \oss and services can be developed and disseminated by various
	architectures that implement this perspective.
		\todo[inline]{Weird phrase, "So consequently?".}
	So consequently, the applications that run on top of them.