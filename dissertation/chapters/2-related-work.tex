\chapter{Related Work}
\label{ch.related-work}

The proposal of this work is related to several other research work.
First, some research papers describing state-of-the-art \lightweight \manycores
processors will be cited. Further, research on different \oss
proposed for such processors will be highlighted.

\section{Lightweight Manycore Processors}
\label{sec.works.manycores}

	In addition to \mppa, many research studies exemplify the wide variety of
	architectural possibilities of \lightweight \manycores.
	For instance, \citeonline{olofsson2014} introduce \epiphany as a
	high-performance energy-efficient \manycore architecture suitable for
	real-time embedded systems.
	The architecture consists of nodes communicating through three 2D mesh \nocs
	with a distributed shared-memory model without coherence protocol.
	Each node has one \risc \cpu, multi-banked local memory, a \dma engine,
	an event monitor and a network interface.
	The three \nocs are independent, scalable, and implement a packet-switched
	model with four duplex links at every node.

	\citeonline{Wallentowitz2013} presents the open-source \optimsoc framework
	for help and facilitate on the manycore processor design. The \optimsoc
	allows to simulate a manycore on a computer or synthesize it on an \fpga.
	It groups \openrisc processors~\footnote{https://opencores.org/openrisc} in tiles.
	The tiles communicate through a \textit{packet-switched \noc}.
	The \noc support various network topologies, depending only on the tiles organization.
	Precisely, a \textit{network adapter} handles the memory transfers between
	tile and the local memory and provides a message-passing communication
	model among tiles.
	The tiles organization and the network topology allow handling
	communication by (i) message-exchange, (ii) partitioned global
	address space without cache coherence, and (iii) global memory
	with cache coherence via a write-through policy.

	Similarly, \citeonline{Kurth2017} introduces the \hero, which
	unites an \arm host processor with a fully modifiable
	\riscv \manycore implemented on an \fpga. The \pmca uses a
	multi-cluster design and relies on multi-banked memory, called \spm.
	Multi-channel \dma engines had substituted the data caches.
	Data transfer occurs between a local \spm and all remote \spms or
	with shared global memory. Communication to main memory passes
	through software-managed lightweight \rab. The \rab performs the
	translation of the virtual-to-physical address, likes an \mmu.
	In this way, clusters can share virtual address pointers.
	Besides, exists different designs for a shared instruction
	cache and top-level interconnection such as bus or \noc.

\section{Operating Systems for Manycores}
\label{sec.works.os}

	\citeonline{Baumann2009} proposed a new \os architecture for scalable multicore
	systems, called \multikernel.
	In their perspective, the future of the \oss is on embracing the networked nature
	of the machines based on distributed systems ideas.
	Assuming the cores are independent nodes of a network, they build the traditional
	\os functionalities as fully-featured processes.
	These processes communicate via message-passing and do not share the internal
	structures of the \os.
	The work showed how expensive it is to maintain a state of the \os through
	shared-memory instead of exchanging messages and the subsequent increase of
	the complexity of cache-coherence protocols.
	The \multikernel implementation, named Barelfish, follows three design principles.
	First, \textit{Make all inter-core communication explicit} turns the system
	amenable to human or automated analysis because processes communicate only
	through well-defined interfaces.
	Second, \textit{Make \os structures hardware-neutral} makes the hardware-independent
	code easy to debug, optimize, and facilitates the deployment of \os for new
	processor types, avoiding rework.
	And lastly, \textit{View \os state as replicated instead of shared} improves system
	scalability.

	In \citeonline{Wisniewski2014}, the concept of scalability was pushed
	to the extreme, thinking on \hpc.
	The principal motivation is the creation of an \os that simultaneously supports
	programmability through support \linux \api, and provides a lightweight kernel
	to performance, scalability, and reliability.
	The \os, named \mos, provides as much of the hardware resources as
	possible to the \hpc applications and the \linux kernel component
	acts as a service that provides \linux functionalities.

	Similarly, \citeonline{kluge2014} developed the \moosca.
	With \moosca, they introduce abstractions that are easily composed, called Nodes,
	Channels, and Servers.
	Where Nodes represent execution resources, Channels represent communication
	resources, \eg \noc resources, and lastly, Servers are nodes that provide
	services to client Nodes.
	To meet safety-critical requirements, they partition \manycore and distribute
	replicas of Servers, turning the whole system more predictable.
	However, in order to deal with interferences in shared resources,
	usage policies are introduced to make possible the prediction of system behavior.

	Finally, \citeonline{nightingale2009} presents the Helios \os to
	simplify the process of writing, deploy, and optimize an application across
	heterogeneous cores.
	They use the microkernel model, naming \textit{satellite kernel}, to export
	a uniform and straightforward set of \os abstractions.
	The most important design decisions were to avoid unnecessary remote communication
	by thinking about the penalty they have in \numa domains.
	Also, request the minimum of hardware primitives so that architectures with many
	constraints can be ported.
	Moreover, request the minimum hardware resources to support architectures with little
	computational power or memory constraints.

\section{Discussion}

	\autoref{sec.works.manycores} exemplifies how \manycore architectures can be
	grouped over a common logic perspective.
	They all have one or more logical units distributed and incorporated on clusters.
	The clusters, interconnected through a network, communicate by message-exchange.
	However, due to the domain for which these processors were designed, they end up
	presenting several differences among them at the hardware level.

	Additionally, \autoref{sec.works.os} presents \oss studies that focus
	on the most efficient exploration of manycores processor characteristics.
	Many of them introduce entirely new concepts, reducing the programmability
	and portability of development environments. Some even seek to provide
	\posix interfaces by porting an adapted version of known kernels, but
	this can lead to optimization losses at near-hardware levels.
	However, the \os and communication models presented fit well with the
	distributed nature of manycores.

	In this context, \nanvix \hal deals with the lowest possible layer
	with a focus on aspects that make it challenging to work with a
	specific group of manycores, \ie \lightweight \manycores. The exported
	interfaces sought to group lightweight manycores on a common and effective
	view. Above \hal, services will be developed that seek first and foremost
	to provide greater programmability and portability through a fully-featured
	\posix-compliant \os.
