\chapter{Conclusions}
\label{ch.conclusions}

	Initially, this work presented a historical evolution of processors from single core
    to manycore. By demonstrating the relationship between the growth of the
    number of cores and energy consumption, it was discussed how academia and
    industry began to develop alternatives to alleviate the technological barriers
    that have emerged. However, even new processors that emerge and stand out
    because of their performance and power consumption lack on programmability
    and portability, because of their architectural features, such as hybrid
    programming model, constrained memory subsystems, no cache coherency, and
    heterogeneous configurations. Part of the difficulty stems from the
    incompleteness of existing \oss and runtimes in dealing with severe
    architectural constraints.

	In this work, we present an inter-cluster communication facility designed
    around the main points in the development of an \os for
    \textit{lightweight manycores}. As a basis, we discussed hardware and software
    aspects of parallel and distributed architectures. Different models of \os
    approaches have been presented that can use the communication facility.
    Thus, to provide the essential functionalities for such \oss, three
    communication abstractions have been proposed for the \nanvixhal with the concern of
    providing \qos: \sync, useful to create distributed
    barriers; \mailbox, which provides the exchange of small messages
    with flow control; and \portal, which allows the exchange
    of arbitrary amounts of data between two clusters.

	Another contribution of this work was the communication services for an \os
    based on the microkernel approach (\textit{Nanvix}). These services can multiplex
    the resources exposed by \hal and perform the verification of the parameters for
    each abstraction. In general, these services securely export the communication
    abstractions to the user, benefiting from the non-competition of \os internal
    structures, because of the separation of master and slave responsibilities.

	The contributions of this dissertation are included in the investment of a
    distributed operating system. Several works are underway in this context,
    including a full port of the \mpi and implementation of a distributed paging
    system that will rely on the proposed abstractions. As future work at the \hal level, we will
    seek to properly utilize existing \dma $\mu$threads to perform asynchronous
    submissions and, consequently, to increase the performance of the abstractions. At
    the microkernel level, we will study the virtualization of the structures of
    each service, and improve multiplexing algorithms, enriching the
    programmability of applications. The results present how well-known
    distributed algorithms can be efficiently supported by \nanvixos and
    encourage improvements provided by the proper use of \dma accelerators.