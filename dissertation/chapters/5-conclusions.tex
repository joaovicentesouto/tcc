\chapter{Conclusions}
\label{ch.conclusions}

Initially, this work presented a historical context of multicore
processors to the nowadays.
By demonstrating the relationship between the growth of the number
of core and energy consumption, it was discussed how academia and
industry began to develop alternatives to alleviate the technological
barriers that have emerged.
However, even new processors that emerge and stand out because of
their performance and power consumption,
they lack on programmability and portability, because of their architectural
features, such as hybrid programming model, constrained memory subsystems,
no cache coherency, and heterogeneous configurations.
Part of the difficulty stems from the incompleteness of existing \oss and
runtimes in dealing with severe architectural constraints.

In this work, we present a inter-cluster communication facility
designed around the main points in the development of an \os for \textit{lightweight manycores}.
As a basis, we discussed hardware and software aspects of parallel
and distributed architectures.
Different models of \os approaches have been presented that can
use the communication facility.
Thus, to provide the basic functionalities for such \oss, three
communication abstractions have been proposed for \hal with the
concern of providing \qos.
Among them is the \sync abstraction to create distributed barriers.
The \mailbox abstraction provides the exchange of small messages
with flow control.
So finally, the \portal abstraction allows the exchange of
arbitrary amounts of data between two clusters.

Another contribution of this work was the communication services
for an \os based on the microkernel approach.
These services provide for the multiplexing of the resources
exposed by \hal and the verification of the parameters required
for each abstraction.
In general, these services securely export the communication
abstractions to the user, benefiting from the non-competition
of \os internal structures because of the separation of master
and slave responsibilities.
The results exhibited expected behaviors for all \mpi collective
communication routines. As future work at the \hal level, we will
seek to enable the use of the $\mu$~threads to perform asynchronous
send and, consequently, increasing the performance of abstractions.
At the microkernel level, we will study the virtualisation of the
structures of each service and improves multiplexing algorithms,
enriching the programmability of applications.

\todo[inline]{Remotar quantitativamente os melhores resultados do
trabalho}

\todo[inline]{Ressaltar que o trabalho está inserido em um contexto de
projeto maior, que ainda está em andamento e dizer o que está sendo
feito em cima da sua primitiva. Ex: port do MPI, implementacao
de um sistema de paginacao distribuido usando os resultados
obtidos aqui. Isso irá reforcar a contribuição.}

\todo[inline]{Deixar mais claro que uthreads está relacionado ao DMA e
relembrar o porque é interessante isso do ponto de vista dos
resultados}
