\chapter{Conclusions}
\label{ch.conclusions}

	Initially, this work presented a historical context of multicore processors
    to the nowadays. By demonstrating the relationship between the growth of the
    number of core and energy consumption, it was discussed how academia and
    industry began to develop alternatives to alleviate the technological barriers
    that have emerged. However, even new processors that emerge and stand out
    because of their performance and power consumption, they lack on programmability
    and portability, because of their architectural features, such as hybrid
    programming model, constrained memory subsystems, no cache coherency, and
    heterogeneous configurations. Part of the difficulty stems from the
    incompleteness of existing \oss and runtimes in dealing with severe
    architectural constraints.

	In this work, we present an inter-cluster communication facility designed
    around the main points in the development of an \os for
    \textit{lightweight manycores}. As a basis, we discussed hardware and software
    aspects of parallel and distributed architectures. Different models of \os
    approaches have been presented that can use the communication facility.
    Thus, to provide the essential functionalities for such \oss, three
    communication abstractions have been proposed for \hal with the concern of
    providing \qos. Among them is the \sync abstraction to create distributed
    barriers. The \mailbox abstraction provides the exchange of small messages
    with flow control. So finally, the \portal abstraction allows the exchange
    of arbitrary amounts of data between two clusters.

	Another contribution of this work was the communication services for an \os
    based on the microkernel approach. These services provide for the multiplexing
    of the resources exposed by \hal and the verification of the parameters for
    each abstraction. In general, these services securely export the communication
    abstractions to the user, benefiting from the non-competition of \os internal
    structures because of the separation of master and slave responsibilities.

	The contributions of this dissertation are included in the investment of a
    distributed operating system. Several works are underway in this context,
    including a full port of the \mpi and implementation of a distributed paging
    system using the results obtained. As future work at the \hal level, we will
    seek to properly utilize existing \dma $\mu$threads to perform asynchronous
    submission and, consequently, increase the performance of abstractions. At
    the microkernel level, we will study the virtualization of the structures of
    each service, and improves multiplexing algorithms, enriching the
    programmability of applications. The results present how well known
    distributed algorithms can be efficiently supported by Nanvix OS and
    encourage improvements provided by the proper use of \dma accelerators.