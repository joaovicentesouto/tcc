\chapter{Source Code}
\label{ch:source-code}

\renewcommand{\thefigure}{\thechapter-\arabic{figure}}
\setcounter{figure}{0}

\renewcommand{\thelisting}{\thechapter-\arabic{listing}}
\setcounter{listing}{0}

\section{Nanvix Project Structure}

	The development of a general-purpose distributed operating system for
	lightweight manycores processors, called Nanvix OS, includes the source
	code for this undergraduate dissertation. \nanvixos is the result of an
	open-source, collaborative project made available on the Github platform.
	Since it is not semantically interesting to have only part of the source
	code and it is impossible to insert all OS code into this document, this
	appendix details where find and test the developed code. The Nanvix
	Project is detailed in \autoref{sec.nanvix}.

	Specifically, there is a separate \textit{Github} repository for each abstraction
	layer that is maintained and updated by Nanvix contributors. Submodules,
	supported by the \textit{git} tool, create an implicit dependency hierarchy
	between the Nanvix repositories. Thus, each layer that depends on another
	has guarantees of its operation and is exempt from its implementation.
	Theses guarantees make the codes better manageable, modular, and better
	portable. All repositories contain test sets (validation and fault tests)
	to ensure the correct implementation of exported interfaces on the several
	supported architectures.

	All repositories follow a branch naming convention, where only two
	branches are, in fact, essential. First, the master branch contains
	the most stable version of the system and marks the significant
	releases of Nanvix versions. Second, the unstable branch contains
	the most current version, but there may still exist bugs or required
	parts missing. The other branches are intended to implement new features,
	improve existing ones, or fix bugs. These other branches are merged
	into the unstable branch when passing all regression tests.

	The following sub-sections, ordered by dependency, detail the four
	repositories that contain the source code of this undergraduate dissertation.

	\subsection{Microkernel-Benchmarks Repository}

		The \textit{Microkernel-Benchmarks} implements the micro-benchmarks
		that stimulate communication services. Specifically, there are four
		micro-benchmarks for each data transfer service, detailed in
		\autoref{sec.evaluation-methodology}. In this repository, there
		are also scripts for compiling and running experiments on the
		\mppa platform.

		The developed code version are available at
		\url{https://github.com/joaovicentesouto/microkernel-benchmarks},
		in the \textit{collective-comm-routines branch} or, specifically,
		in the \textit{commit aafd9a70f8188105efabd651050bc7cafc39d343}.
		\autoref{fig:tree.benchs} presents, in the form of a directory
		tree, all files that are fully, or partially, developed. The
		\textit{\texttt{include}} folder contains only one file with static
		experiment settings. The \textit{\texttt{src}} folder is made up of
		auxiliary files and the micro-benchmarks themselves.

		\begin{figure}[!ht]
			\centering%
			\caption{Directory tree with developed source codes in Microkernel-Benchmarks repository.}%
			\label{fig:tree.benchs}%
			\begin{forest}
			for tree={
				grow'=0,
				parent anchor=children,
				child anchor=parent,
				anchor=parent,
			},
			where level=0{
				draw
			}{
				if={(n()==1)&&(level()>1)}{
				calign with current edge
				}{},
				if n children=0{folder}{},
				edge path'={(!u.parent anchor) -- ++(5pt,0) |- (.child anchor)},
			}
			[microkernel-benchmarks
				[include
					[kbench.h]
				]
				[src
					[utils
						[args.c]
						[barrier.c]
						[crt0.c]
						[node.c]
						[results.c]
						[string.c]
						[times.c]
					]
					[mailbox
						[allgather
							[main.c]
						]
						[broadcast
							[main.c]
						]
						[gather
							[main.c]
						]
						[pingpong
							[main.c]
						]
					]
					[portal
						[allgather
							[main.c]
						]
						[broadcast
							[main.c]
						]
						[gather
							[main.c]
						]
						[pingpong
							[main.c]
						]
					]
				]
			]
			\end{forest}%
			\fonte{Developed by the author.}%
		\end{figure}

	\subsection{LibNanvix Repository}

		LibNanvix defines and exports user interfaces for Nanvix Microkernel
		services. Implementations, in turn, have the responsibility of making
		requests to the master core through the kernel call interface exported
		by the microkernel repository, detailed in \autoref{sec.microkernel}.
		Briefly, this repository is the complement of the Microkernel repository.

		\autoref{fig:tree.libnanvix} illustrates, as a directory tree,
		all files developed of the communication abstraction user
		interface. First, the files in the \textit{\texttt{include}} folder
		define the user interfaces. Second, files within the
		\textit{\texttt{ikc}} folder perform checks to identify potential
		problems. Finally, the test folder contains the validation
		and correctness tests of user abstractions. The code version
		is available in \textit{commit a9dcb35dd8727aefe41d316ac2609c88073e160e}
		at \url{https://github.com/nanvix/libnanvix}.

		\begin{figure}[!ht]
			\centering%
			\caption{Directory tree with developed source codes in LibNanvix repository.}%
			\label{fig:tree.libnanvix}%
			\begin{forest}
			for tree={
				grow'=0,
				parent anchor=children,
				child anchor=parent,
				anchor=parent,
			},
			where level=0{
				draw
			}{
				if={(n()==1)&&(level()>1)}{
				calign with current edge
				}{},
				if n children=0{folder}{},
				edge path'={(!u.parent anchor) -- ++(5pt,0) |- (.child anchor)},
			}
			[libnanvix
				[include/nanvix/sys
					[mailbox.h]
					[portal.h]
					[sync.h]
					[noc.h]
				]
				[src
					[libnanvix/ikc
						[mailbox.c]
						[portal.c]
						[sync.c]
						[noc.c]
					]
					[test
						[kmailbox.c]
						[kportal.c]
						[ksync.c]
						[knoc.c]
					]
				]
			]
			\end{forest}%
			\fonte{Developed by the author.}%
		\end{figure}

	\subsection{Microkernel Repository}

		Microkernel, in the context of the asymmetric microkernel, covers the
		responsibilities of the master core. This repository contains all
		internal microkernel structures, manipulation functions, and
		master-side kernel calls. In association with LibNanvix, they provide
		the skeletons of system abstractions within the cluster of a lightweight
		manycores. For more details, see \autoref{sec.microkernel}.

		\autoref{fig:tree.microkernel} exemplifies the three developed file sets.
		First, the \textit{\texttt{include}} folder defines static kernel configurations;
		exports internal call interfaces, i.e., executed by the master; and
		external calls, i.e., the kernel call system. Second, the files within
		the \textit{\texttt{noc}} folder integrate the internal structures and proper
		protection, manipulation, and multiplexing functions. Finally, the
		\textit{\texttt{sys}} folder contains the implementation of kernel calls.
		The code version
		is available in \textit{commit a9826dec62baa3fe47ab3a77b15f3ccfdd84b79a}
		at \url{https://github.com/nanvix/microkernel}.

		\begin{figure}[!ht]
			\centering%
			\caption{Directory tree with developed source codes in Microkernel repository.}%
			\label{fig:tree.microkernel}%
			\begin{forest}
			for tree={
				grow'=0,
				parent anchor=children,
				child anchor=parent,
				anchor=parent,
			},
			where level=0{
				draw
			}{
				if={(n()==1)&&(level()>1)}{
				calign with current edge
				}{},
				if n children=0{folder}{},
				edge path'={(!u.parent anchor) -- ++(5pt,0) |- (.child anchor)},
			}
			[microkernel
				[include/nanvix
					[syscall.h]
					[mailbox.h]
					[portal.h]
					[sync.h]
				]
				[src/kernel
					[noc
						[mailbox.c]
						[portal.c]
						[sync.c]
					]
					[sys
						[syscall.c]
						[mailbox.c]
						[portal.c]
						[sync.c]
						[noc.c]
					]
				]
			]
			\end{forest}%
			\fonte{Developed by the author.}%
		\end{figure}

	\subsection{Hardware Abstraction Layer (HAL) Repository}

		\hal defines and exports the lowest-level abstraction interfaces, detailed
		in \autoref{sec.hal}. This repository deals directly with multiple supported
		architectures, currently including \mppa, \optimsoc, and \hero.
		An implementation on a \unix \os has also been developed to allow easy
		debugging of the developed systems.

		\autoref{fig:tree.hal} illustrates the logical structure designed to
		facilitate \hal portability. In this framework, there are levels that
		abstract the hardware layers from the essential element (\textit{core})
		to the level that abstracts the architecture as a whole (\textit{target}).
		Expressly, this work was restricted to \textit{processor} and
		\textit{target} levels. The file hierarchy follows what is described in
		the previous sub-sections, dividing the files into interfaces and
		implementations. The code version is available in
		\textit{commit 1e7d3bc64decff023ac91cdecc2e0ac6c53ac946} at
		\url{https://github.com/nanvix/hal}.

		The interfaces are included in the include folder. However, there is a
		distinction between exported interfaces and architecture-specific
		interfaces. The \textit{\texttt{nanvix/hal}} folder covers interfaces exported
		to other repositories, as well as static checking of specific interfaces.
		The \textit{\texttt{hal/arch}} folder actually exports what architecture has
		hardware capabilities. If it does not implement a hardware limitation
		feature, a default implementation will be used. For each hardware
		interface defined in the \textit{\texttt{hal/arch}}, there is a corresponding
		source file (\textit{\texttt{src}} folder). Specific implementations of \mppa
		architecture, for example, contain implementations of low-level
		communications, including handling and manipulation of different
		existing \nocs. Finally, the test folder constitutes the integration
		tests that validate the various implementations.

		\begin{figure}[!ht]
			\centering%
			\caption{Directory tree with developed source codes in HAL repository.}%
			\label{fig:tree.hal}%
			\begin{forest}
			for tree={
				grow'=0,
				parent anchor=children,
				child anchor=parent,
				anchor=parent,
			},
			where level=0{
				draw
			}{
				if={(n()==1)&&(level()>1)}{
				calign with current edge
				}{},
				if n children=0{folder}{},
				edge path'={(!u.parent anchor) -- ++(5pt,0) |- (.child anchor)},
			}
			[hal
				[include
					[nanvix/hal
							[processor
								[noc.h]
								[clusters.h]
							]
							[target
								[mailbox.h]
								[portal.h]
								[sync.h]
							]
					]
					[hal/arch
						[processor/bostan
							[noc.h]
							[clusters.h]
							[noc
								[tag.h]
								[ctag.h]
								[dtag.h]
								[dma.h]
							]
						]
						[target/kalray/mppa256
							[mailbox.h]
							[portal.h]
							[sync.h]
						]
					]
				]
				[src
					[hal/arch
							[processor/bostan
								[dma.c]
								[noc.c]
								[ctag.c]
								[dtag.c]
								[clusters.c]
							]
							[target/mppa256
								[mailbox.c]
								[portal.c]
								[sync.c]
							]
					]
					[test
						[processor
							[noc.c]
							[cnoc.c]
							[dnoc.c]
							[clusters.c]
						]
						[target
							[mailbox.c]
							[portal.c]
							[sync.c]
						]
					]
				]
			]
			\end{forest}%
			\fonte{Developed by the author.}%
		\end{figure}

\section{Regression Testing Example}
\label{sec:code-example}

	\autoref{code:regression-tests} exemplifies the running of user interfaces
	regression tests on the \mppa platform. A server that contains an installed
	\mppa card compiles and executes the tests remotely. The tests can still
	run locally using the \textit{Docker} tool, where a virtual machine
	simulates one of the architectures supported by Nanvix OS.

\begin{listing}[!hb]
\caption{Bash script for regression testing on the MPPA-256 platform.}
\label{code:regression-tests}
\begin{minted}{bash}
#!/bin/bash
# A Shell Script To Runs Regression Tests on MPPA-256
# Nanvix Project - 2019

# Defines the target architecture.
export TARGET=mppa256

# Download the LibNanvix repository.
git clone https://github.com/nanvix/libnanvix.git

# Enter the folder containing the source code.
cd libnanvix

# Switch to the version developed in this work.
git checkout a9dcb35dd8727aefe41d316ac2609c88073e160e

# Download submodules recursively.
git submodule update --init --recursive

# Compile the dependencies.
make contrib

# Compile, link to libraries, and create the executable.
make all

# Runs the regression tests.
make test
\end{minted}
\fonte{Developed by the author.}
\end{listing}