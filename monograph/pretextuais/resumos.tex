% resumo em inglês
\begin{resumo}[brazil]
Em conjunto com a maior escalabilidade de desempenho e eficiência energética,
os \textit{lightweight manycores} trouxeram um novo conjunto de desafios no
desenvolvimento de software, provenientes de suas particularidades arquitetônicas.
Nesse contexto, os sistemas operacionais são essenciais porque permitem abstrair
as características do hardware sob uma perspectiva simplificada e produtiva.
Assim, os sistemas operacionais tornam o desenvolvimento de aplicativos
menos dispendioso e mais eficiente.
No entanto, parte dos desafios encontrados em \textit{lightweight manycores}
deriva dos runtimes existentes e dos sistemas operacionais que não lidam
completamente com as complexidades existentes.
Assim, acreditamos que os sistemas operacionais para a próxima geração de
\textit{lightweight manycores} devem ser reprojetados do zero para lidar
com suas severas restrições arquitetônicas.
Em particular, devido à natureza distribuída dos \textit{manycores}, as
abstrações de comunicação desempenham um papel crucial na escalabilidade
e desempenho desses processadores.
Neste cenário, o objetivo deste trabalho é desenvolver um módulo de comunicação
entre clusters para o processador \textit{manycore} emergente MPPA-256.
Este módulo faz parte de um HAL genérico e flexível para \textit{lightweight manycores}
que lida com os principais problemas encontrados no projeto de um sistema operacional
para esses processadores.
Encima desse módulo, serviços de comunicação também serão propostos para
um sistema operacional baseado no modelo \textit{microkernel}, que busca
fornecer um esqueleto básico para as abstrações do sistema.
Essas contribuições estão inseridas em um contexto de pesquisa mais amplo
que procura investigar um sistema operacional distribuído completo baseado
em uma abordagem multikernel para esses processadores.
\end{resumo}

\begin{resumo}[english]

Jointly with further performance scalability and energy efficiency,
lightweight manycores brought a new set of challenges in software
development coming from their architectural particularities.
In this context, \oses are essential because they allow to abstract
the characteristics of the hardware under a simplified and productive
perspective.
Thus, \oses make application development less costly and more efficient.
However, part of the challenges encountered in lightweight manycores
derives from existing runtimes and \oses that do not completely handle
extant intricacies.
So, we believe that \oses for the next-generation of lightweight
manycores must be redesigned from scratch to cope with their
tight architectural constraints.
In particular, due to the distributed nature of the manycores,
communication abstractions play a crucial role in the scalability
and performance of these processors.
In this scenario, the goal of this work is to develop an inter-cluster
communication module for the emergent \mppa Lightweight Manycore Processor.
This module is part of a generic and flexible \hal for lightweight
manycores that cope with the key issues encountered in designing
an \os for these processors.
On top of the module, communication services will also be proposed
for a microkernel-based \os that seeks to provide bare bones for
system abstractions.
These contributions are embedded in a broader research context
that seeks to investigate a fully-featured distributed \os based
on a multikernel approach to these processors.

\end{resumo}