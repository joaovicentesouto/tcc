% resumo em inglês
\begin{resumo}[brazil]
Resumo aqui.
\end{resumo}

\begin{resumo}[english]
% - Where do manycores from?
    For some years now, computer systems grew the core parallelism and improved diverse other architectural aspects to soften the impact of the frequency barrier achieved, looking for continuous scale performance.
    However, for supercomputers to reach processing power (FLOPS) in the \exascale (10$^{18}$), computer systems must take into account their energy consumption~\cite{darpa:exascale}. Therefore, a new architecture category of processors, denominated \textit{lightweight} \manycores, emerged to provide high parallelism with low-power consumption.

% - Manycores characteristics
    These new processors
    (i) integrate thousands of low-power cores in a single die; 
    (ii) are designed to cope with \mimd workloads;
    (iii) rely on a high-bandwidth \noc for fast and reliable message-passing communication;
    (iv) present constrained memory systems; and
    (v) frequently feature a heterogeneous configuration.
    Some industry-successful examples of lightweight manycores are
    the \mppa~\cite{DeDinechin2013-1};
    the \epiphany~\cite{Olofsson2014}; and
    the \taihulight~\cite{Zheng2015}.

% Motivation
% - Why is development on manycores difficult?
    Jointly with further performance scalability and energy efficiency, manycores brought a new set of challenges in software development coming from their architectural particularities.
    Precisely, these particularities force developers 
    (i) to adopt a message-passing programming model; 
    (ii) to deal explicitly with cache coherency;
    (iii) to take data tiling and prefetching approach for handling the small local memory and multiple address spaces; and
    (iv) to manage the complexity of development to commonly heterogeneous architecture.
% - Why develop for their processors?
    

% Challenges and Problem Definition
% - Why these problems exist? (Because the existing OSes does not handle architecture particularities)
% - These particularities prevent common OSes that easy portated without a complex redesign. And existing OSes does not account some architectural points.

% Goals and Contributions
% - Redesign from scratch around all their tight architectural constraints.
% - Focus on addressing first-order programmability challenges
% - Introducing generic and flexible HAL for lightweight manycores
\end{resumo}