\chapter{Related Works}
\label{ch.related-works}

The proposal of this work is related to several other research works
on lightweight \manycores.
First, some research papers describing state-of-the-art \manycores
processors will be cited. Further, research on different operating
systems proposed for such processors will be highlighted.

\section{State-of-the-art of Manycore Processors}

	Olofsson \etal~\cite{olofsson2014}, introduce Epiphany as a high-performance energy-efficient \manycore
	architecture suitable for real-time embedded systems.
	The architecture consists of nodes connected to a 2D mesh \noc with a distributed shared-memory model
	without coherence protocol.
	Each node has one \risc \cpu, multi-banked local memory, a \dma engine, an event monitor and a network interface.
	The network interface consists of three networks where one is used for reading request and the other two are used
	for write transactions destined for on-chip and off-chip.

	% DFMC 2015
	% On another hand, Zheng \etal~\cite{zheng2015} presents a heterogeneous \manycore processor
	% named \dfmc for high performance computing systems.
	% \dfmc integrates computing processing elements clusters, management processing elements and
	% memory controllers which heterogeneous processor cores. Using a unified execution model,
	% \dfmc able a share-memory with suporting to cache coherence by 

	On another hand, for help and facilitate on the \manycore processor design, Wallentowitz \etal~\cite{Wallentowitz2013}
	presents the open-source framework \optimsoc which allows build \manycore \soc and simulate them on a computer
	or synthesize them on a \fpga.
	The \pes are \openrisc~\footnote{https://opencores.org/openrisc} processor organized in tiles.
	The central architectural element is the LISNoC, a \textit{packet-switched \noc} that implements virtual channels
	to avoid message-dependent deadlocks.
	The LISNoC support various network topologies depending only on the tiles organization.
	Specifically, a \textit{network adapter} handles the memory transfers between tile and the memory and provide
	hardware means to a message-passing communication model among tiles.

	Similarly, Kurth \etal~\cite{Kurth2017} introduce the hero which unites an ARM Cortex-A host processor with a
	fully modifiable \risc-V \manycore implemented on a \fpga.
	The \pmca uses a multi-cluster design em relies on multi-banked, \spms.
	The data caches had substituted to a multi-channel \dma engine that copy data between a shared L1 \spm and
	remote \smps or shared main memory.
	In addition, exists different designs for the shared instruction caches and top-level interconnection such as bus or \noc.

\section{Operating Systems for Lightweight \manycore Processors}

	Popcorn 2015
	Multikernel 2009
	Moosca 2014
	mOS 2014

\section{Discussion}
	In this section, I discuss ...