\chapter{Schedule}
\label{ch.schedule}

This chapter presents the schedule for the next activities
planned for the development of the undergraduate dissertation.

\section{Activities}
\label{sec:gantt}

\begin{figure}[!h]
	\caption{Chart Gantt of the Schedule.}

	\begin{center}
		\begin{ganttchart}[
			x unit=0.6cm,
			y unit title=0.6cm,
			y unit chart=0.6cm,
			hgrid,
			vgrid={{dotted, dotted, dotted, dotted, dotted, dotted}},
			% title label font=\3scriptsize,
			title/.append style={fill=gray!30},
			title height=1,
			bar/.append style={fill=gray!30,rounded corners=2pt},
			bar label font=\scriptsize,
			group label font=\scriptsize,
		]{7}{12}

		\gantttitle{\textbf{Meses}}{6} \\
		\gantttitle{\textbf{2019}}{6} \\
		\gantttitlelist{7,8,9,10,11,12}{1} \\
		\ganttbar{1. Writing the Implementation and Experiments.}{7}{9} \\
		\ganttbar{2. In-depth Writing of the Proposal.}{9}{10} \\
		\ganttbar{3. Presentation of the Undergraduate Dissertation.}{11}{11} \\
		\ganttbar{4. Review and Final Submission of the Undergraduate Dissertation.}{12}{12} \\

		\end{ganttchart}
	\end{center}

	\label{chart.gantt}
\end{figure}

	Figure \ref{chart.gantt} shows the planned activities and their durations visually.
	Beginning in July 2019, the final submission is planned for December 2019.
	In detail, the activities are described below:

	\begin{itemize}
		\item \textit{Writing the Implementation and Experiments:}
			Currently, the inter-cluster communication module has the \sync abstraction
			completed and part of the \mailbox abstraction.
			Since the \portal abstraction uses the same low-level mechanisms of the others
			abstractions, its implementation will be facilitated.
			The communication services already have prototypes developed by the author
			for a symmetric \os.
			In this way, the prototypes will need to be modified to use the \hal and
			modified for a master-slave model.
			Finally, micro-benchmarks will be developed to perform an analysis of the
			performance of the implemented services.
		\item \textit{In-depth Writing of the Proposal:}
			During September and October, this activity will be committed to improving
			this draft and detailing the project decisions chose and implementations produced.
			We will also describe the experiments performed and discuss the results obtained.
		\item \textit{Presentation of the Undergraduate Dissertation:}
			November will be dedicated to the development and preparation of the presentation
			of the work and the results achieved.
			So finally, to present the dissertation to the evaluators.
		\item \textit{Review and Final Submission of the Undergraduate Dissertation:}
			Finally, December will be dedicated to the correction of the issues indicated
			by the evaluators and finalized with the final submission of the dissertation.
	\end{itemize}

 \chapter{Conclusions}
\label{ch.conclusions}

Iniciamente, este trabalho apresentou um contexto histórico
dos processadores multicores até os tempos atuais.
Ao demonstrar a relação entre o aumento de núcleos e o consumo
de energia, foi discutido como a academia e a industria
começaram a desenvolver alternativas para amenizar as
barreiras tecnológicas que surgiram.
Contudo, mesmo os novos processadores que surgiram
se destacarem por causa do seu desempenho e consumo energético,
eles pecam em programabilidade e portabilidade proveniente
das suas características arquiteturais, tais como, modelo de
programação híbrido, subsistemas de memória restritivos,
falta de coerência de cache e configurações heterogênicas.
Parte das dificuldades deriva da incompletudo dos sistemas operacionais
e runtimes existentes em lidar com as severas restrições arquitetônicas.

Neste trabalho será desenvolvido um módulo de comunicação entre cluster
projetado em torno dos principais pontos no desenvolvimento de um
sistema operacional.
Como base, foi discutido aspectos de hardware e softwares
de arquiteturas paralelas e distribuidas.
Foram apresentadas modelos distintos de abordagens de sistemas operacionais
que podem a vir utilizar o módulo de comunicação.
Deste modo, para fornecer o esqueleto básico para tais sistemas operacionais,
três abstrações de comunicação foram propostas para a HAL com a preocupação
de prover qualidade de serviço.
Entre elas estão a abstração sync para criar barreiras distibuidas.
A abstração mailbox fornece a troca de mensagens pequenas com controle 
de fluxo.
E por fim, a abstração portal possibilita a troca de quantidade arbitrárias
de dados entre dois clusters.

Outra contribuição deste trabalho apresentada foram os 
serviços de comunicação para um sistema operacional baseado na abordagem microkernel.
Esses serviços providenciam a multiplexação dos recursos expostos pela HAL
e a verificação dos parâmetros necessários para cada abstração.
De forma geral, esses serviços exportam uma forma segura ao usuário
se beneficiando da não concorrência das estruturas internas do SO 
por causa da separação de responsabilidades de mestre e escravo.
Por fim, a proposta detalhou, de forma geral, vários aspectos
das implementações.
Devido ao fato dos serviços dependerem do módulo de comunicação da HAL
para serem desenvolvido, os tópicos associados ao módulo acabaram sendo
mais detalhados e melhor explorados.
Entretanto, a próxima versão da dissertação especificará com maior clareza
e objetividade ambas as contribuições deste trabalho.
