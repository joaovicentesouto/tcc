\chapter{Schedule}
\label{ch.schedule}

This chapter presents the schedule for the next activities
planned for the development of the undergraduate dissertation.

\section{Activities}
\label{sec:gantt}

\begin{figure}[!h]
	\caption{Chart Gantt of the Schedule.}

	\begin{center}
		\begin{ganttchart}[
			x unit=0.6cm,
			y unit title=0.6cm,
			y unit chart=0.6cm,
			hgrid,
			vgrid={{dotted, dotted, dotted, dotted, dotted, dotted}},
			% title label font=\3scriptsize,
			title/.append style={fill=gray!30},
			title height=1,
			bar/.append style={fill=gray!30,rounded corners=2pt},
			bar label font=\scriptsize,
			group label font=\scriptsize,
		]{7}{12}

		\gantttitle{\textbf{Meses}}{6} \\
		\gantttitle{\textbf{2019}}{6} \\
		\gantttitlelist{7,8,9,10,11,12}{1} \\
		\ganttbar{1. Writing the Implementation and Experiments.}{7}{9} \\
		\ganttbar{2. In-depth Writing of the Proposal.}{9}{10} \\
		\ganttbar{3. Presentation of the Undergraduate Dissertation.}{11}{11} \\
		\ganttbar{4. Review and Final Submission of the Undergraduate Dissertation.}{12}{12} \\

		\end{ganttchart}
	\end{center}

	\label{chart.gantt}
\end{figure}

	Figure \ref{chart.gantt} shows the planned activities and their durations visually.
	Beginning in July 2019, the final submission is planned for December 2019.
	In detail, the activities are described below:

	\begin{itemize}
		\item \textit{Writing the Implementation and Experiments:}
			Currently, the inter-cluster communication module has the \sync abstraction
			completed and part of the \mailbox abstraction.
			Since the \portal abstraction uses the same low-level mechanisms of the others
			abstractions, its implementation will be facilitated.
			The communication services already have prototypes developed by the author
			for a symmetric \os.
			In this way, the prototypes will need to be modified to use the \hal and
			modified for a master-slave model.
			Finally, micro-benchmarks will be developed to perform an analysis of the
			performance of the implemented services.
		\item \textit{In-depth Writing of the Proposal:}
			During September and October, this activity will be committed to improving
			this draft and detailing the project decisions chose and implementations produced.
			We will also describe the experiments performed and discuss the results obtained.
		\item \textit{Presentation of the Undergraduate Dissertation:}
			November will be dedicated to the development and preparation of the presentation
			of the work and the results achieved.
			So finally, to present the dissertation to the evaluators.
		\item \textit{Review and Final Submission of the Undergraduate Dissertation:}
			Finally, December will be dedicated to the correction of the issues indicated
			by the evaluators and finalized with the final submission of the dissertation.
	\end{itemize}

 \chapter{Conclusions}
\label{ch.conclusions}

Initially, this work presented a historical context of multicore
processors to the nowadays.
By demonstrating the relationship between the growth of the number
of core and energy consumption, it was discussed how academia and
industry began to develop alternatives to alleviate the technological
barriers that have emerged.
However, even new processors that emerge and stand out because of
their performance and power consumption,
they sin in programmability and portability because of their architectural
features, such as hybrid programming model, restrictive memory subsystems,
lack of cache coherence, and heterogeneous configurations.
Part of the difficulty stems from the incompleteness of existing \oses and
runtimes in dealing with severe architectural constraints.

In this work, we present a communication module between the cluster
designed around the main points in the development of an \os for \textit{lightweight manycores}.
As a basis, we discussed hardware and software aspects of parallel
and distributed architectures.
Different models of \os approaches have been presented that can
use the communication module.
Thus, to provide the basic framework for such \oses, three
communication abstractions have been proposed for \hal with the
concern of providing quality of service.
Among them is the \sync abstraction to create distributed barriers.
The \mailbox abstraction provides the exchange of small messages
with flow control.
So finally, the \portal abstraction allows the exchange of
arbitrary amounts of data between two clusters.

Another contribution of this work was the communication services
for an operating system based on the microkernel approach.
These services provide for the multiplexing of the resources
exposed by \hal and the verification of the parameters required
for each abstraction.
In general, these services securely export the communication
abstractions to the user, benefiting from the non-competition
of \os internal structures because of the separation of master
and slave responsibilities.
Lastly, the proposal detailed, in general, several aspects of
the implementations.
Because the communication services depend on the \hal communication
module to be developed, the topics associated with
the module have become more detailed and better explored.
However, the next version of the undergraduate dissertation
will clearly and objectively specify both contributions of this work.
