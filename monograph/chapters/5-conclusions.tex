\chapter{Schedule}
\label{ch.schedule}

\todo[inline]{Descrever textualmente as atividades}

% \begin{tabularx}{\linewidth}{|X|*{6}{c|}}
% 	\hline
% 	\multicolumn{1}{|c|}{\multirow{2}{*}{Etapas}} & \multicolumn{6}{|c|}{Meses}\\ \cline{2-7}
% 	& jan & fev & mar & abr & maio & jun \\ \hline

% 	Estudar conceitos de Sistemas Operacionais
% 	&  x  &  x  &     &     &     &     \\ \hline

% 	Estudar Componentes e Serviços existentes
% 	&     &  x  &  x  &     &     &     \\ \hline

% 	Projetar os Componentes e Serviços para o microkernel
% 	&     &     &  x  &  x  &  x  &  x  \\ \hline

% 	Desenvolver implementação
% 	&     &     &     &  x  &  x  &  x  \\ \hline

% 	Testar e validar a implementação
% 	&     &     &     &     &  x  &  x  \\ \hline

% 	Documentar o aprendizado
% 	&  x  &  x  &  x  &  x  &  x  &  x  \\ \hline

% \end{tabularx}

%%%% Modelo com barras %%%%

Exemplo:

\begin{figure}[!h]
	\begin{center}

		\begin{ganttchart}[
			x unit=0.7cm,
			y unit title=0.7cm,
			y unit chart=0.7cm,
			hgrid,
			vgrid={{dotted, dotted, dotted, dotted, dotted, dotted}},
			% title label font=\3scriptsize,
			title/.append style={fill=gray!30},
			title height=1,
			bar/.append style={fill=gray!30,rounded corners=2pt},
			bar label font=\scriptsize,
			group label font=\scriptsize,
		]{1}{6}

		\gantttitle{\textbf{Meses}}{6} \\
		\gantttitle{\textbf{2019}}{6} \\
		\gantttitlelist{1,2,3,4,5,6}{1} \\
		\ganttbar{1. Estudar conceitos de Sistemas Operacionais}{1}{2} \\
		\ganttbar{2. Estudar Componentes e Serviços existentes}{2}{3} \\
		\ganttbar{2. Projetar os Componentes e Serviços para o microkernel}{3}{6} \\
		\ganttbar{3. Desenvolver implementação}{4}{6} \\
		\ganttbar{3. Testar e validar a implementação}{5}{6} \\
		\ganttbar{4. Documentar o aprendido}{1}{6} \\

		\end{ganttchart}
	\end{center}
	\label{tab:cronograma}
\end{figure}

 \chapter{Conclusions}
\label{ch.conclusions}

\todo[inline]{4 ou 5 parágrafos para sumarizar o que foi apresentado.}