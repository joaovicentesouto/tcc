\chapter{Proposal}
\label{ch.proposal}

\todo[inline]{Adicionar introdução ao capítulo, sumarizando o que será apresentado aqui. Indicar que as duas principais contribuições serão no ``inter-cluster communication module'' (o qual será portado para o MPPA) e nos ``communication services'' (que é mais genérico e poderá ser usado para outras plataformas).}

\section{Inter-Cluster Communication Module}

    % Parte do desenvolvimento
    Em todas as abstrações sempre existem dois papeis que são assumidos por um
    ou mais clusters dependendo em qual dos lados da operação o cluster está.
    O cluster nomeado de "ONE" irá sempre assumir o papel de receptor de sinais
    ou dados de outros clusters.
    Os clusters emissores são nomeados de "ALL".


\subsection{Sync}

\todo[inline]{Quais são as funções? Como pretendes implementá-las? Quais são os recursos (API ou hardware) do MPPA que serão explorados para implementar essa interface?}

\subsection{Mailbox}

\todo[inline]{Quais são as funções? Como pretendes implementá-las? Quais são os recursos (API ou hardware) do MPPA que serão explorados para implementar essa interface?}

\subsection{Portal}

\todo[inline]{Quais são as funções? Como pretendes implementá-las? Quais são os recursos (API ou hardware) do MPPA que serão explorados para implementar essa interface?}

\section{Communication Services}

\todo[inline]{Quais são os serviços? Como pretendes implementá-los? Quais são os recursos (API ou hardware) do MPPA que serão explorados para implementar esses serviços?}
