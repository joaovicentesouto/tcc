\chapter{Theoretical Fundamentation} % Or Fundamentals?
\label{ch.fundamentation}

	In this chapter, I introduce the fundamentals concepts related to the present work.

\section{Operating Systems Concepts}
	This section may come sooner (here).

\section{Multiple Processor Systems}
\label{sec.multiple_processor_systems}

	According to Tanenbaum~\cite{tanenbaum:4ed}, exists three models of
	modern multiple processor architectures.
	A shared-memory multiprocessor, a message-passing multicomputer, and a
	wide area distributed systems.
	The sections below address the two first models presenting significant
	hardware and software concepts for the present monograph.

	\subsection{Multiprocessors}

		In the early days of electronic digital computing, John Von Neumann proposed
		an architectural model for computers to be easily programmable~\cite{von-neumann:model}.
		As shown in Figure X, this model describes a Central Processing Unit that
		loads the instructions and data from a Memory Unit, dealing with inputs
		and generating outputs to I/O Devices.
		Modern processors still follow this model, but some components and behaviors
		are specialized or replicated to increase performance.
		
		In this context, a shared-memory multiprocessor is a computer system in which
		two or more CPUs share full access to a common RAM~\cite{tanenbaum:4ed}.
		Concurrency issues begin to appear where are many CPUs competing for shared resources.
		For instance, many threads of a process running on different CPUs can loose
		or overwrite values of a shared variable. So, low-level software, like
		Operating Systems, needs to handle those issues and provide management systems
		to user-level.
		Below sections will first present an overview of multiprocessor hardware and
		then move on to Operating Systems' issues.

		\subsubsection{Multiprocessors Hardware}

		Hardware overview.
		Von Neumann Model -> Main Systems Organization
		- Maybe speak UMA and NUMA.
		- Flynn's Organization.

		\subsubsection{Multiprocessors Software}

		Software overview.
		- OS
		- Processor, thread (simiraly with Flynn Organization)


	\subsection{Multicomputers}
		Hardware overview.
		Software overview.

		Important: Low-level Software and User-level Software primarily for communication.

	\subsection{Manycores}
		Focus on manycores.

		Use the above concepts to build the narrative on manycores.

	\subsubsection{MPPA-256}
		Focus on manycores.

\section{Multikernel OS Concept}
	In this section, I will write about the Multikernel OS concept using Nanvix has an example.

\section{Microkernel OS Concept}
	Focus on microkernel.

\section{Nanvix HAL}
	Focus on HAL.