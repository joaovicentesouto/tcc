\chapter{Theoretical Fundamentation} % Or Fundamentals?
\label{ch.fundamentation}

    In this chapter, I introduce the fundamentals concepts related to the present work.

\section{Operating Systems Concepts}
    This section may come sooner (here).

\section{Multiple Processor Systems}

    According to Tanenbaum, exists three models of modern multiple processor architectures.
    A shared-memory multiprocessor, a message-passing multicomputer, and a wide area distributed systems.
    The sections below address the two first models presenting significant hardware and software concepts for the present monograph.

    \subsection{Multiprocessors}
    Hardware overview.
    Von Neumann Model -> Main Systems Organization
    - Maybe speak UMA and NUMA.
    - Flynn's Organization.
        
    Software overview.
    - OS
    - Processor, thread (simiraly with Flynn Organization)


\subsection{Multicomputers}
    Hardware overview.
    Software overview.

    Important: Low-level Software and User-level Software primarily for communication.

\subsection{Manycores}
    Focus on manycores.

    Use the above concepts to build the narrative on manycores.

\subsubsection{MPPA-256}
    Focus on manycores.

\section{Multikernel OS Concept}
    In this section, I will write about the Multikernel OS concept using Nanvix has an example.

\section{Microkernel OS Concept}
    Focus on microkernel.

\section{Nanvix HAL}
    Focus on HAL.